%%% -*-LaTeX-*-
%%% ====================================================================
%%% This file is intended to be included as a front-matter section of
%%% the sample thesis, with a command like this
%%%
%%% \optionalfront{Typesetting Experiments}{%%% -*-LaTeX-*-
%%% ====================================================================
%%% This file is intended to be included as a front-matter section of
%%% the sample thesis, with a command like this
%%%
%%% \optionalfront{Typesetting Experiments}{%%% -*-LaTeX-*-
%%% ====================================================================
%%% This file is intended to be included as a front-matter section of
%%% the sample thesis, with a command like this
%%%
%%% \optionalfront{Typesetting Experiments}{%%% -*-LaTeX-*-
%%% ====================================================================
%%% This file is intended to be included as a front-matter section of
%%% the sample thesis, with a command like this
%%%
%%% \optionalfront{Typesetting Experiments}{\input{samples}}
%%%
%%% just before the \maintext macro that begins the body of the thesis,
%%% and starts chapter numbering.
%%%
%%% [16-Mar-2016]
%%% ====================================================================

\input{rgb.sty}

\definecolor{utahred} {rgb} {0.8, 0.0, 0.0} % official definition for University of Utah Printing Services

\doublespace

%% \showthe \baselineskip

In this section, we use color in several places.
The \verb=\colorbox= command takes two arguments
--- a named color and text to be in black on a
background of that color --- and sets the text in
a box with a small margin of width \verb=\fboxsep=
(set to \texttt{\the\fboxsep} in this document).

Here, we want a tighter colored box that has a
fixed height, and is independent of letter shape.
We set the margin to zero inside a group so that
the change is purely local, and so that height and
depth of the line are not increased over what they
would be if the colored box were not used.  We
prefix a \TeX{} \verb=\strut= to the user-supplied
text, because that command expands to a zero-width
box of the height and depth of parentheses, which,
in most fonts, delimit the extent of letter
shapes.

\begin{verbatim}
\newcommand {\hilitebox} [1] {{\fboxsep = 0pt\colorbox{pink}{\strut #1}}}
\end{verbatim}

\newcommand {\hilitebox} [1] {{\fboxsep = 0pt\colorbox{pink}{\strut #1}}}

Here is a fragment from the first chapter in another
thesis, set in \emph{emphasized text} to distinguish it
from the rest of this section:

\begin{itshape}
    In light of the known results, the consistency
    of empirical semivariogram and related
    estimators is widely considered a settled
    matter.  For example, Lahiri, Lee, and Cressie
    \cite{Lahiri:2002:ADA} state:

    \begin{quote}
        The simpler and more commonly used
        nonparametric estimators of the variogram,
        such as the method of moments estimator of
        Matheron (1962) and its robustified
        versions due to Cressie and Hawkins (1980)
        have many desirable properties like,
        unbiasedness, consistency, etc. \ldots
    \end{quote}
    \noindent
    Regarding a kernel estimator of the covariance
    function, Hall and Patil
    \cite{Hall:1994:PNE} remarked:
    \begin{quote}
        It is not difficult to see that if, as $ n
        $ increases, the points $ t_i $ become
        increasingly dense in each bounded subset
        of $ \mathbb{R}^d $, then the bandwidth $
        h $ may be chosen so that $ \check \rho(t)
        \to \rho(t) $ as $ n \to \infty $, for
        each $ t \in \mathbb{R}^d $.
    \end{quote}
    However, in order to be true, such statements
    would need to be qualified by many assumptions
    on the random field as well as on the
    observation locations. We will see in
    \S2.3 that even for well-behaved
    random fields (e.g., $\rho^*$-mixing Gaussian
    random fields), it is not enough to assume
    that the observation locations become
    increasingly dense in each bounded subset; a
    stronger assumption must be made to ensure
    that the observation locations do not become
    denser in one region too much faster than in
    others.
\end{itshape}

The text before the previous paragraph contained
two \texttt{quote} environments separated by a
line of prose.  Here are some more tests of both
kinds of \LaTeX{} environments for showing text
written by someone else.

This is a \hilitebox{\texttt{quote}} environment
with one short line, following a fairly short
paragraph of prose (in this, and following
examples, the text is explicitly colored with a
command like \verb=\color{purple}= inside the
environment before the text):
%
\begin{singlespace}
\color{darkblue}
\begin{verbatim}
\begin{quote}
    \color{purple}
    14 March 2016 is $\pi \approx 3.1416$ day in funny notation.
    \hfill \emph{Web news reports}
\end{quote}
\end{verbatim}
\end{singlespace}
%
\begin{quote}
    \color{purple}
    14 March 2016 is $\pi \approx 3.1416$ day in funny notation.
    \hfill \emph{Web news reports}
\end{quote}

This is a \hilitebox{\texttt{quote}} environment
with three short lines, each a separate paragraph,
following a fairly short paragraph of prose.

\begin{singlespace}
\color{darkblue}
\begin{verbatim}
\begin{quote}
    \color{forestgreen}
    14 March 2016 is $\pi \approx 3.1416$ day in funny notation.
    \hfill \emph{Web news reports}

    14 March 2016 is $\pi \approx 3.1416$ day in funny notation.
    \hfill \emph{Web news reports}

    14 March 2016 is $\pi \approx 3.1416$ day in funny notation.
    \hfill \emph{Web news reports}
\end{quote}
\end{verbatim}
\end{singlespace}
%
\begin{quote}
    \color{forestgreen}
    14 March 2016 is $\pi \approx 3.1416$ day in funny notation.
    \hfill \emph{Web news reports}

    14 March 2016 is $\pi \approx 3.1416$ day in funny notation.
    \hfill \emph{Web news reports}

    14 March 2016 is $\pi \approx 3.1416$ day in funny notation.
    \hfill \emph{Web news reports}
\end{quote}

Here is another example, this time with separate
colors for each paragraph:

\begin{singlespace}
\color{darkblue}
\begin{verbatim}
\begin{quote}
    \color{darkkhaki}
    14 March 2016 is $\pi \approx 3.1416$ day in funny notation.
    \hfill \emph{Web news reports}

    \color{darkmagenta}
    14 March 2016 is $\pi \approx 3.1416$ day in funny notation.
    \hfill \emph{Web news reports}

    \color{darkcyan}
    14 March 2016 is $\pi \approx 3.1416$ day in funny notation.
    \hfill \emph{Web news reports}

    \color{darkorange}
    14 March 2016 is $\pi \approx 3.1416$ day in funny notation.
    14 March 2016 is $\pi \approx 3.1416$ day in funny notation.
    14 March 2016 is $\pi \approx 3.1416$ day in funny notation.
    \linebreak
    \strut
    \hfill \emph{Web news reports}
\end{quote}
\end{verbatim}
\end{singlespace}
%
\begin{quote}
    \color{darkkhaki}
    14 March 2016 is $\pi \approx 3.1416$ day in funny notation.
    \hfill \emph{Web news reports}

    \color{darkmagenta}
    14 March 2016 is $\pi \approx 3.1416$ day in funny notation.
    \hfill \emph{Web news reports}

    \color{darkcyan}
    14 March 2016 is $\pi \approx 3.1416$ day in funny notation.
    \hfill \emph{Web news reports}

    \color{darkorange}
    14 March 2016 is $\pi \approx 3.1416$ day in funny notation.
    14 March 2016 is $\pi \approx 3.1416$ day in funny notation.
    14 March 2016 is $\pi \approx 3.1416$ day in funny notation.
    \linebreak
    \strut
    \hfill \emph{Web news reports}
\end{quote}

Notice that \hilitebox{\texttt{quote}} paragraphs
are \emph{not} indented, but the environment
itself \emph{is} indented on the left and right by
the value of \verb=\leftmargin= (set to
\texttt{\the\leftmargin} in this document, which
should be identical to \verb=2.5em=, where
\verb=1em= = \texttt{\dimen0 = 1em \the\dimen0}).

For debugging purposes, we also have
\verb=\leftmargini= set to
\texttt{\the\leftmargini}, and we have
\verb=\leftmarginii= set to
\texttt{\the\leftmarginii}.

This is a \hilitebox{\texttt{quotation}}
environment with one paragraph, following a fairly
short paragraph of prose (notice that the
\texttt{quotation} paragraphs \emph{are}
indented):

\begin{singlespace}
\color{darkblue}
\begin{verbatim}
\begin{quotation}
    \color{blue}
    Algebra is concerned with manipulation in
    \emph{time}, and geometry is concerned with
    \emph{space}. These are two orthogonal aspects
    of the world, and they represent two different
    points of view in mathematics.  Thus the
    argument or dialogue between mathematicians in
    the past about the relative importance of
    geometry and algebra represents something very
    fundamental.
    \hfill
    \emph{Sir Michael Atiyah}
    % Mathematics in the 20$^{th}$ century
    % NTM {\bf 10}(1--3) 25--39 (September 2002)
    % http://dx.doi.org/10.1007/BF03033096
\end{quotation}
\end{verbatim}
\end{singlespace}
%
\begin{quotation}
    \color{blue}
    Algebra is concerned with manipulation in
    \emph{time}, and geometry is concerned with
    \emph{space}. These are two orthogonal aspects
    of the world, and they represent two different
    points of view in mathematics.  Thus the
    argument or dialogue between mathematicians in
    the past about the relative importance of
    geometry and algebra represents something very
    fundamental.
    \hfill
    \emph{Sir Michael Atiyah}
    % Mathematics in the 20$^{th}$ century
    % NTM {\bf 10}(1--3) 25--39 (September 2002)
    % http://dx.doi.org/10.1007/BF03033096
\end{quotation}

This is a \hilitebox{\texttt{quotation}} environment with three
paragraphs, following a fairly short paragraph of prose:

\begin{quotation}
    \color{utahred}
    Algebra is concerned with manipulation in
    \emph{time}, and geometry is concerned with
    \emph{space}. These are two orthogonal aspects
    of the world, and they represent two different
    points of view in mathematics.  Thus the
    argument or dialogue between mathematicians in
    the past about the relative importance of
    geometry and algebra represents something very
    fundamental.
    \hfill
    \emph{Sir Michael Atiyah}

    Algebra is concerned with manipulation in
    \emph{time}, and geometry is concerned with
    \emph{space}. These are two orthogonal aspects
    of the world, and they represent two different
    points of view in mathematics.  Thus the
    argument or dialogue between mathematicians in
    the past about the relative importance of
    geometry and algebra represents something very
    fundamental.
    \hfill
    \emph{Sir Michael Atiyah}

    Algebra is concerned with manipulation in
    \emph{time}, and geometry is concerned with
    \emph{space}. These are two orthogonal aspects
    of the world, and they represent two different
    points of view in mathematics.  Thus the
    argument or dialogue between mathematicians in
    the past about the relative importance of
    geometry and algebra represents something very
    fundamental.
    \hfill
    \emph{Sir Michael Atiyah}
\end{quotation}

%% \showthe \baselineskip

Now all following text should be back in
double-spaced mode, and just go on and on and on
and on and on and on and on and on and on and on
and on and on and on and on and on and on and on
and on and on and on and on and on and on and on
and on and on and on and on and on and on and on
and on and on and on and on and on and on and on
and on and on and on and on and on and on and on
and on and on and on and on and on and on and on
and on and on and on and on and on and on and on
and on and on and on and on and on and on and on
and on and on and on and on and on and on and on
and on and on and on and on \ldots{}.

%% \showthe \baselineskip

Now all following text should be back in
double-spaced mode, and just go on and on and on
and on and on and on and on and on and on and on
and on and on and on and on and on and on and on
and on and on and on and on and on and on and on
and on and on and on and on and on and on and on
and on and on and on and on and on and on and on
and on and on and on and on and on and on and on
and on and on and on and on and on and on and on
and on and on and on and on and on and on and on
and on and on and on and on and on and on and on
and on and on and on and on and on and on and on
and on and on and on and on \ldots{}.

%%% \showthe \baselineskip
}
%%%
%%% just before the \maintext macro that begins the body of the thesis,
%%% and starts chapter numbering.
%%%
%%% [16-Mar-2016]
%%% ====================================================================

\input{rgb.sty}

\definecolor{utahred} {rgb} {0.8, 0.0, 0.0} % official definition for University of Utah Printing Services

\doublespace

%% \showthe \baselineskip

In this section, we use color in several places.
The \verb=\colorbox= command takes two arguments
--- a named color and text to be in black on a
background of that color --- and sets the text in
a box with a small margin of width \verb=\fboxsep=
(set to \texttt{\the\fboxsep} in this document).

Here, we want a tighter colored box that has a
fixed height, and is independent of letter shape.
We set the margin to zero inside a group so that
the change is purely local, and so that height and
depth of the line are not increased over what they
would be if the colored box were not used.  We
prefix a \TeX{} \verb=\strut= to the user-supplied
text, because that command expands to a zero-width
box of the height and depth of parentheses, which,
in most fonts, delimit the extent of letter
shapes.

\begin{verbatim}
\newcommand {\hilitebox} [1] {{\fboxsep = 0pt\colorbox{pink}{\strut #1}}}
\end{verbatim}

\newcommand {\hilitebox} [1] {{\fboxsep = 0pt\colorbox{pink}{\strut #1}}}

Here is a fragment from the first chapter in another
thesis, set in \emph{emphasized text} to distinguish it
from the rest of this section:

\begin{itshape}
    In light of the known results, the consistency
    of empirical semivariogram and related
    estimators is widely considered a settled
    matter.  For example, Lahiri, Lee, and Cressie
    \cite{Lahiri:2002:ADA} state:

    \begin{quote}
        The simpler and more commonly used
        nonparametric estimators of the variogram,
        such as the method of moments estimator of
        Matheron (1962) and its robustified
        versions due to Cressie and Hawkins (1980)
        have many desirable properties like,
        unbiasedness, consistency, etc. \ldots
    \end{quote}
    \noindent
    Regarding a kernel estimator of the covariance
    function, Hall and Patil
    \cite{Hall:1994:PNE} remarked:
    \begin{quote}
        It is not difficult to see that if, as $ n
        $ increases, the points $ t_i $ become
        increasingly dense in each bounded subset
        of $ \mathbb{R}^d $, then the bandwidth $
        h $ may be chosen so that $ \check \rho(t)
        \to \rho(t) $ as $ n \to \infty $, for
        each $ t \in \mathbb{R}^d $.
    \end{quote}
    However, in order to be true, such statements
    would need to be qualified by many assumptions
    on the random field as well as on the
    observation locations. We will see in
    \S2.3 that even for well-behaved
    random fields (e.g., $\rho^*$-mixing Gaussian
    random fields), it is not enough to assume
    that the observation locations become
    increasingly dense in each bounded subset; a
    stronger assumption must be made to ensure
    that the observation locations do not become
    denser in one region too much faster than in
    others.
\end{itshape}

The text before the previous paragraph contained
two \texttt{quote} environments separated by a
line of prose.  Here are some more tests of both
kinds of \LaTeX{} environments for showing text
written by someone else.

This is a \hilitebox{\texttt{quote}} environment
with one short line, following a fairly short
paragraph of prose (in this, and following
examples, the text is explicitly colored with a
command like \verb=\color{purple}= inside the
environment before the text):
%
\begin{singlespace}
\color{darkblue}
\begin{verbatim}
\begin{quote}
    \color{purple}
    14 March 2016 is $\pi \approx 3.1416$ day in funny notation.
    \hfill \emph{Web news reports}
\end{quote}
\end{verbatim}
\end{singlespace}
%
\begin{quote}
    \color{purple}
    14 March 2016 is $\pi \approx 3.1416$ day in funny notation.
    \hfill \emph{Web news reports}
\end{quote}

This is a \hilitebox{\texttt{quote}} environment
with three short lines, each a separate paragraph,
following a fairly short paragraph of prose.

\begin{singlespace}
\color{darkblue}
\begin{verbatim}
\begin{quote}
    \color{forestgreen}
    14 March 2016 is $\pi \approx 3.1416$ day in funny notation.
    \hfill \emph{Web news reports}

    14 March 2016 is $\pi \approx 3.1416$ day in funny notation.
    \hfill \emph{Web news reports}

    14 March 2016 is $\pi \approx 3.1416$ day in funny notation.
    \hfill \emph{Web news reports}
\end{quote}
\end{verbatim}
\end{singlespace}
%
\begin{quote}
    \color{forestgreen}
    14 March 2016 is $\pi \approx 3.1416$ day in funny notation.
    \hfill \emph{Web news reports}

    14 March 2016 is $\pi \approx 3.1416$ day in funny notation.
    \hfill \emph{Web news reports}

    14 March 2016 is $\pi \approx 3.1416$ day in funny notation.
    \hfill \emph{Web news reports}
\end{quote}

Here is another example, this time with separate
colors for each paragraph:

\begin{singlespace}
\color{darkblue}
\begin{verbatim}
\begin{quote}
    \color{darkkhaki}
    14 March 2016 is $\pi \approx 3.1416$ day in funny notation.
    \hfill \emph{Web news reports}

    \color{darkmagenta}
    14 March 2016 is $\pi \approx 3.1416$ day in funny notation.
    \hfill \emph{Web news reports}

    \color{darkcyan}
    14 March 2016 is $\pi \approx 3.1416$ day in funny notation.
    \hfill \emph{Web news reports}

    \color{darkorange}
    14 March 2016 is $\pi \approx 3.1416$ day in funny notation.
    14 March 2016 is $\pi \approx 3.1416$ day in funny notation.
    14 March 2016 is $\pi \approx 3.1416$ day in funny notation.
    \linebreak
    \strut
    \hfill \emph{Web news reports}
\end{quote}
\end{verbatim}
\end{singlespace}
%
\begin{quote}
    \color{darkkhaki}
    14 March 2016 is $\pi \approx 3.1416$ day in funny notation.
    \hfill \emph{Web news reports}

    \color{darkmagenta}
    14 March 2016 is $\pi \approx 3.1416$ day in funny notation.
    \hfill \emph{Web news reports}

    \color{darkcyan}
    14 March 2016 is $\pi \approx 3.1416$ day in funny notation.
    \hfill \emph{Web news reports}

    \color{darkorange}
    14 March 2016 is $\pi \approx 3.1416$ day in funny notation.
    14 March 2016 is $\pi \approx 3.1416$ day in funny notation.
    14 March 2016 is $\pi \approx 3.1416$ day in funny notation.
    \linebreak
    \strut
    \hfill \emph{Web news reports}
\end{quote}

Notice that \hilitebox{\texttt{quote}} paragraphs
are \emph{not} indented, but the environment
itself \emph{is} indented on the left and right by
the value of \verb=\leftmargin= (set to
\texttt{\the\leftmargin} in this document, which
should be identical to \verb=2.5em=, where
\verb=1em= = \texttt{\dimen0 = 1em \the\dimen0}).

For debugging purposes, we also have
\verb=\leftmargini= set to
\texttt{\the\leftmargini}, and we have
\verb=\leftmarginii= set to
\texttt{\the\leftmarginii}.

This is a \hilitebox{\texttt{quotation}}
environment with one paragraph, following a fairly
short paragraph of prose (notice that the
\texttt{quotation} paragraphs \emph{are}
indented):

\begin{singlespace}
\color{darkblue}
\begin{verbatim}
\begin{quotation}
    \color{blue}
    Algebra is concerned with manipulation in
    \emph{time}, and geometry is concerned with
    \emph{space}. These are two orthogonal aspects
    of the world, and they represent two different
    points of view in mathematics.  Thus the
    argument or dialogue between mathematicians in
    the past about the relative importance of
    geometry and algebra represents something very
    fundamental.
    \hfill
    \emph{Sir Michael Atiyah}
    % Mathematics in the 20$^{th}$ century
    % NTM {\bf 10}(1--3) 25--39 (September 2002)
    % http://dx.doi.org/10.1007/BF03033096
\end{quotation}
\end{verbatim}
\end{singlespace}
%
\begin{quotation}
    \color{blue}
    Algebra is concerned with manipulation in
    \emph{time}, and geometry is concerned with
    \emph{space}. These are two orthogonal aspects
    of the world, and they represent two different
    points of view in mathematics.  Thus the
    argument or dialogue between mathematicians in
    the past about the relative importance of
    geometry and algebra represents something very
    fundamental.
    \hfill
    \emph{Sir Michael Atiyah}
    % Mathematics in the 20$^{th}$ century
    % NTM {\bf 10}(1--3) 25--39 (September 2002)
    % http://dx.doi.org/10.1007/BF03033096
\end{quotation}

This is a \hilitebox{\texttt{quotation}} environment with three
paragraphs, following a fairly short paragraph of prose:

\begin{quotation}
    \color{utahred}
    Algebra is concerned with manipulation in
    \emph{time}, and geometry is concerned with
    \emph{space}. These are two orthogonal aspects
    of the world, and they represent two different
    points of view in mathematics.  Thus the
    argument or dialogue between mathematicians in
    the past about the relative importance of
    geometry and algebra represents something very
    fundamental.
    \hfill
    \emph{Sir Michael Atiyah}

    Algebra is concerned with manipulation in
    \emph{time}, and geometry is concerned with
    \emph{space}. These are two orthogonal aspects
    of the world, and they represent two different
    points of view in mathematics.  Thus the
    argument or dialogue between mathematicians in
    the past about the relative importance of
    geometry and algebra represents something very
    fundamental.
    \hfill
    \emph{Sir Michael Atiyah}

    Algebra is concerned with manipulation in
    \emph{time}, and geometry is concerned with
    \emph{space}. These are two orthogonal aspects
    of the world, and they represent two different
    points of view in mathematics.  Thus the
    argument or dialogue between mathematicians in
    the past about the relative importance of
    geometry and algebra represents something very
    fundamental.
    \hfill
    \emph{Sir Michael Atiyah}
\end{quotation}

%% \showthe \baselineskip

Now all following text should be back in
double-spaced mode, and just go on and on and on
and on and on and on and on and on and on and on
and on and on and on and on and on and on and on
and on and on and on and on and on and on and on
and on and on and on and on and on and on and on
and on and on and on and on and on and on and on
and on and on and on and on and on and on and on
and on and on and on and on and on and on and on
and on and on and on and on and on and on and on
and on and on and on and on and on and on and on
and on and on and on and on and on and on and on
and on and on and on and on \ldots{}.

%% \showthe \baselineskip

Now all following text should be back in
double-spaced mode, and just go on and on and on
and on and on and on and on and on and on and on
and on and on and on and on and on and on and on
and on and on and on and on and on and on and on
and on and on and on and on and on and on and on
and on and on and on and on and on and on and on
and on and on and on and on and on and on and on
and on and on and on and on and on and on and on
and on and on and on and on and on and on and on
and on and on and on and on and on and on and on
and on and on and on and on and on and on and on
and on and on and on and on \ldots{}.

%%% \showthe \baselineskip
}
%%%
%%% just before the \maintext macro that begins the body of the thesis,
%%% and starts chapter numbering.
%%%
%%% [16-Mar-2016]
%%% ====================================================================

\input{rgb.sty}

\definecolor{utahred} {rgb} {0.8, 0.0, 0.0} % official definition for University of Utah Printing Services

\doublespace

%% \showthe \baselineskip

In this section, we use color in several places.
The \verb=\colorbox= command takes two arguments
--- a named color and text to be in black on a
background of that color --- and sets the text in
a box with a small margin of width \verb=\fboxsep=
(set to \texttt{\the\fboxsep} in this document).

Here, we want a tighter colored box that has a
fixed height, and is independent of letter shape.
We set the margin to zero inside a group so that
the change is purely local, and so that height and
depth of the line are not increased over what they
would be if the colored box were not used.  We
prefix a \TeX{} \verb=\strut= to the user-supplied
text, because that command expands to a zero-width
box of the height and depth of parentheses, which,
in most fonts, delimit the extent of letter
shapes.

\begin{verbatim}
\newcommand {\hilitebox} [1] {{\fboxsep = 0pt\colorbox{pink}{\strut #1}}}
\end{verbatim}

\newcommand {\hilitebox} [1] {{\fboxsep = 0pt\colorbox{pink}{\strut #1}}}

Here is a fragment from the first chapter in another
thesis, set in \emph{emphasized text} to distinguish it
from the rest of this section:

\begin{itshape}
    In light of the known results, the consistency
    of empirical semivariogram and related
    estimators is widely considered a settled
    matter.  For example, Lahiri, Lee, and Cressie
    \cite{Lahiri:2002:ADA} state:

    \begin{quote}
        The simpler and more commonly used
        nonparametric estimators of the variogram,
        such as the method of moments estimator of
        Matheron (1962) and its robustified
        versions due to Cressie and Hawkins (1980)
        have many desirable properties like,
        unbiasedness, consistency, etc. \ldots
    \end{quote}
    \noindent
    Regarding a kernel estimator of the covariance
    function, Hall and Patil
    \cite{Hall:1994:PNE} remarked:
    \begin{quote}
        It is not difficult to see that if, as $ n
        $ increases, the points $ t_i $ become
        increasingly dense in each bounded subset
        of $ \mathbb{R}^d $, then the bandwidth $
        h $ may be chosen so that $ \check \rho(t)
        \to \rho(t) $ as $ n \to \infty $, for
        each $ t \in \mathbb{R}^d $.
    \end{quote}
    However, in order to be true, such statements
    would need to be qualified by many assumptions
    on the random field as well as on the
    observation locations. We will see in
    \S2.3 that even for well-behaved
    random fields (e.g., $\rho^*$-mixing Gaussian
    random fields), it is not enough to assume
    that the observation locations become
    increasingly dense in each bounded subset; a
    stronger assumption must be made to ensure
    that the observation locations do not become
    denser in one region too much faster than in
    others.
\end{itshape}

The text before the previous paragraph contained
two \texttt{quote} environments separated by a
line of prose.  Here are some more tests of both
kinds of \LaTeX{} environments for showing text
written by someone else.

This is a \hilitebox{\texttt{quote}} environment
with one short line, following a fairly short
paragraph of prose (in this, and following
examples, the text is explicitly colored with a
command like \verb=\color{purple}= inside the
environment before the text):
%
\begin{singlespace}
\color{darkblue}
\begin{verbatim}
\begin{quote}
    \color{purple}
    14 March 2016 is $\pi \approx 3.1416$ day in funny notation.
    \hfill \emph{Web news reports}
\end{quote}
\end{verbatim}
\end{singlespace}
%
\begin{quote}
    \color{purple}
    14 March 2016 is $\pi \approx 3.1416$ day in funny notation.
    \hfill \emph{Web news reports}
\end{quote}

This is a \hilitebox{\texttt{quote}} environment
with three short lines, each a separate paragraph,
following a fairly short paragraph of prose.

\begin{singlespace}
\color{darkblue}
\begin{verbatim}
\begin{quote}
    \color{forestgreen}
    14 March 2016 is $\pi \approx 3.1416$ day in funny notation.
    \hfill \emph{Web news reports}

    14 March 2016 is $\pi \approx 3.1416$ day in funny notation.
    \hfill \emph{Web news reports}

    14 March 2016 is $\pi \approx 3.1416$ day in funny notation.
    \hfill \emph{Web news reports}
\end{quote}
\end{verbatim}
\end{singlespace}
%
\begin{quote}
    \color{forestgreen}
    14 March 2016 is $\pi \approx 3.1416$ day in funny notation.
    \hfill \emph{Web news reports}

    14 March 2016 is $\pi \approx 3.1416$ day in funny notation.
    \hfill \emph{Web news reports}

    14 March 2016 is $\pi \approx 3.1416$ day in funny notation.
    \hfill \emph{Web news reports}
\end{quote}

Here is another example, this time with separate
colors for each paragraph:

\begin{singlespace}
\color{darkblue}
\begin{verbatim}
\begin{quote}
    \color{darkkhaki}
    14 March 2016 is $\pi \approx 3.1416$ day in funny notation.
    \hfill \emph{Web news reports}

    \color{darkmagenta}
    14 March 2016 is $\pi \approx 3.1416$ day in funny notation.
    \hfill \emph{Web news reports}

    \color{darkcyan}
    14 March 2016 is $\pi \approx 3.1416$ day in funny notation.
    \hfill \emph{Web news reports}

    \color{darkorange}
    14 March 2016 is $\pi \approx 3.1416$ day in funny notation.
    14 March 2016 is $\pi \approx 3.1416$ day in funny notation.
    14 March 2016 is $\pi \approx 3.1416$ day in funny notation.
    \linebreak
    \strut
    \hfill \emph{Web news reports}
\end{quote}
\end{verbatim}
\end{singlespace}
%
\begin{quote}
    \color{darkkhaki}
    14 March 2016 is $\pi \approx 3.1416$ day in funny notation.
    \hfill \emph{Web news reports}

    \color{darkmagenta}
    14 March 2016 is $\pi \approx 3.1416$ day in funny notation.
    \hfill \emph{Web news reports}

    \color{darkcyan}
    14 March 2016 is $\pi \approx 3.1416$ day in funny notation.
    \hfill \emph{Web news reports}

    \color{darkorange}
    14 March 2016 is $\pi \approx 3.1416$ day in funny notation.
    14 March 2016 is $\pi \approx 3.1416$ day in funny notation.
    14 March 2016 is $\pi \approx 3.1416$ day in funny notation.
    \linebreak
    \strut
    \hfill \emph{Web news reports}
\end{quote}

Notice that \hilitebox{\texttt{quote}} paragraphs
are \emph{not} indented, but the environment
itself \emph{is} indented on the left and right by
the value of \verb=\leftmargin= (set to
\texttt{\the\leftmargin} in this document, which
should be identical to \verb=2.5em=, where
\verb=1em= = \texttt{\dimen0 = 1em \the\dimen0}).

For debugging purposes, we also have
\verb=\leftmargini= set to
\texttt{\the\leftmargini}, and we have
\verb=\leftmarginii= set to
\texttt{\the\leftmarginii}.

This is a \hilitebox{\texttt{quotation}}
environment with one paragraph, following a fairly
short paragraph of prose (notice that the
\texttt{quotation} paragraphs \emph{are}
indented):

\begin{singlespace}
\color{darkblue}
\begin{verbatim}
\begin{quotation}
    \color{blue}
    Algebra is concerned with manipulation in
    \emph{time}, and geometry is concerned with
    \emph{space}. These are two orthogonal aspects
    of the world, and they represent two different
    points of view in mathematics.  Thus the
    argument or dialogue between mathematicians in
    the past about the relative importance of
    geometry and algebra represents something very
    fundamental.
    \hfill
    \emph{Sir Michael Atiyah}
    % Mathematics in the 20$^{th}$ century
    % NTM {\bf 10}(1--3) 25--39 (September 2002)
    % http://dx.doi.org/10.1007/BF03033096
\end{quotation}
\end{verbatim}
\end{singlespace}
%
\begin{quotation}
    \color{blue}
    Algebra is concerned with manipulation in
    \emph{time}, and geometry is concerned with
    \emph{space}. These are two orthogonal aspects
    of the world, and they represent two different
    points of view in mathematics.  Thus the
    argument or dialogue between mathematicians in
    the past about the relative importance of
    geometry and algebra represents something very
    fundamental.
    \hfill
    \emph{Sir Michael Atiyah}
    % Mathematics in the 20$^{th}$ century
    % NTM {\bf 10}(1--3) 25--39 (September 2002)
    % http://dx.doi.org/10.1007/BF03033096
\end{quotation}

This is a \hilitebox{\texttt{quotation}} environment with three
paragraphs, following a fairly short paragraph of prose:

\begin{quotation}
    \color{utahred}
    Algebra is concerned with manipulation in
    \emph{time}, and geometry is concerned with
    \emph{space}. These are two orthogonal aspects
    of the world, and they represent two different
    points of view in mathematics.  Thus the
    argument or dialogue between mathematicians in
    the past about the relative importance of
    geometry and algebra represents something very
    fundamental.
    \hfill
    \emph{Sir Michael Atiyah}

    Algebra is concerned with manipulation in
    \emph{time}, and geometry is concerned with
    \emph{space}. These are two orthogonal aspects
    of the world, and they represent two different
    points of view in mathematics.  Thus the
    argument or dialogue between mathematicians in
    the past about the relative importance of
    geometry and algebra represents something very
    fundamental.
    \hfill
    \emph{Sir Michael Atiyah}

    Algebra is concerned with manipulation in
    \emph{time}, and geometry is concerned with
    \emph{space}. These are two orthogonal aspects
    of the world, and they represent two different
    points of view in mathematics.  Thus the
    argument or dialogue between mathematicians in
    the past about the relative importance of
    geometry and algebra represents something very
    fundamental.
    \hfill
    \emph{Sir Michael Atiyah}
\end{quotation}

%% \showthe \baselineskip

Now all following text should be back in
double-spaced mode, and just go on and on and on
and on and on and on and on and on and on and on
and on and on and on and on and on and on and on
and on and on and on and on and on and on and on
and on and on and on and on and on and on and on
and on and on and on and on and on and on and on
and on and on and on and on and on and on and on
and on and on and on and on and on and on and on
and on and on and on and on and on and on and on
and on and on and on and on and on and on and on
and on and on and on and on and on and on and on
and on and on and on and on \ldots{}.

%% \showthe \baselineskip

Now all following text should be back in
double-spaced mode, and just go on and on and on
and on and on and on and on and on and on and on
and on and on and on and on and on and on and on
and on and on and on and on and on and on and on
and on and on and on and on and on and on and on
and on and on and on and on and on and on and on
and on and on and on and on and on and on and on
and on and on and on and on and on and on and on
and on and on and on and on and on and on and on
and on and on and on and on and on and on and on
and on and on and on and on and on and on and on
and on and on and on and on \ldots{}.

%%% \showthe \baselineskip
}
%%%
%%% just before the \maintext macro that begins the body of the thesis,
%%% and starts chapter numbering.
%%%
%%% [16-Mar-2016]
%%% ====================================================================

\input{rgb.sty}

\definecolor{utahred} {rgb} {0.8, 0.0, 0.0} % official definition for University of Utah Printing Services

\doublespace

%% \showthe \baselineskip

In this section, we use color in several places.
The \verb=\colorbox= command takes two arguments
--- a named color and text to be in black on a
background of that color --- and sets the text in
a box with a small margin of width \verb=\fboxsep=
(set to \texttt{\the\fboxsep} in this document).

Here, we want a tighter colored box that has a
fixed height, and is independent of letter shape.
We set the margin to zero inside a group so that
the change is purely local, and so that height and
depth of the line are not increased over what they
would be if the colored box were not used.  We
prefix a \TeX{} \verb=\strut= to the user-supplied
text, because that command expands to a zero-width
box of the height and depth of parentheses, which,
in most fonts, delimit the extent of letter
shapes.

\begin{verbatim}
\newcommand {\hilitebox} [1] {{\fboxsep = 0pt\colorbox{pink}{\strut #1}}}
\end{verbatim}

\newcommand {\hilitebox} [1] {{\fboxsep = 0pt\colorbox{pink}{\strut #1}}}

Here is a fragment from the first chapter in another
thesis, set in \emph{emphasized text} to distinguish it
from the rest of this section:

\begin{itshape}
    In light of the known results, the consistency
    of empirical semivariogram and related
    estimators is widely considered a settled
    matter.  For example, Lahiri, Lee, and Cressie
    \cite{Lahiri:2002:ADA} state:

    \begin{quote}
        The simpler and more commonly used
        nonparametric estimators of the variogram,
        such as the method of moments estimator of
        Matheron (1962) and its robustified
        versions due to Cressie and Hawkins (1980)
        have many desirable properties like,
        unbiasedness, consistency, etc. \ldots
    \end{quote}
    \noindent
    Regarding a kernel estimator of the covariance
    function, Hall and Patil
    \cite{Hall:1994:PNE} remarked:
    \begin{quote}
        It is not difficult to see that if, as $ n
        $ increases, the points $ t_i $ become
        increasingly dense in each bounded subset
        of $ \mathbb{R}^d $, then the bandwidth $
        h $ may be chosen so that $ \check \rho(t)
        \to \rho(t) $ as $ n \to \infty $, for
        each $ t \in \mathbb{R}^d $.
    \end{quote}
    However, in order to be true, such statements
    would need to be qualified by many assumptions
    on the random field as well as on the
    observation locations. We will see in
    \S2.3 that even for well-behaved
    random fields (e.g., $\rho^*$-mixing Gaussian
    random fields), it is not enough to assume
    that the observation locations become
    increasingly dense in each bounded subset; a
    stronger assumption must be made to ensure
    that the observation locations do not become
    denser in one region too much faster than in
    others.
\end{itshape}

The text before the previous paragraph contained
two \texttt{quote} environments separated by a
line of prose.  Here are some more tests of both
kinds of \LaTeX{} environments for showing text
written by someone else.

This is a \hilitebox{\texttt{quote}} environment
with one short line, following a fairly short
paragraph of prose (in this, and following
examples, the text is explicitly colored with a
command like \verb=\color{purple}= inside the
environment before the text):
%
\begin{singlespace}
\color{darkblue}
\begin{verbatim}
\begin{quote}
    \color{purple}
    14 March 2016 is $\pi \approx 3.1416$ day in funny notation.
    \hfill \emph{Web news reports}
\end{quote}
\end{verbatim}
\end{singlespace}
%
\begin{quote}
    \color{purple}
    14 March 2016 is $\pi \approx 3.1416$ day in funny notation.
    \hfill \emph{Web news reports}
\end{quote}

This is a \hilitebox{\texttt{quote}} environment
with three short lines, each a separate paragraph,
following a fairly short paragraph of prose.

\begin{singlespace}
\color{darkblue}
\begin{verbatim}
\begin{quote}
    \color{forestgreen}
    14 March 2016 is $\pi \approx 3.1416$ day in funny notation.
    \hfill \emph{Web news reports}

    14 March 2016 is $\pi \approx 3.1416$ day in funny notation.
    \hfill \emph{Web news reports}

    14 March 2016 is $\pi \approx 3.1416$ day in funny notation.
    \hfill \emph{Web news reports}
\end{quote}
\end{verbatim}
\end{singlespace}
%
\begin{quote}
    \color{forestgreen}
    14 March 2016 is $\pi \approx 3.1416$ day in funny notation.
    \hfill \emph{Web news reports}

    14 March 2016 is $\pi \approx 3.1416$ day in funny notation.
    \hfill \emph{Web news reports}

    14 March 2016 is $\pi \approx 3.1416$ day in funny notation.
    \hfill \emph{Web news reports}
\end{quote}

Here is another example, this time with separate
colors for each paragraph:

\begin{singlespace}
\color{darkblue}
\begin{verbatim}
\begin{quote}
    \color{darkkhaki}
    14 March 2016 is $\pi \approx 3.1416$ day in funny notation.
    \hfill \emph{Web news reports}

    \color{darkmagenta}
    14 March 2016 is $\pi \approx 3.1416$ day in funny notation.
    \hfill \emph{Web news reports}

    \color{darkcyan}
    14 March 2016 is $\pi \approx 3.1416$ day in funny notation.
    \hfill \emph{Web news reports}

    \color{darkorange}
    14 March 2016 is $\pi \approx 3.1416$ day in funny notation.
    14 March 2016 is $\pi \approx 3.1416$ day in funny notation.
    14 March 2016 is $\pi \approx 3.1416$ day in funny notation.
    \linebreak
    \strut
    \hfill \emph{Web news reports}
\end{quote}
\end{verbatim}
\end{singlespace}
%
\begin{quote}
    \color{darkkhaki}
    14 March 2016 is $\pi \approx 3.1416$ day in funny notation.
    \hfill \emph{Web news reports}

    \color{darkmagenta}
    14 March 2016 is $\pi \approx 3.1416$ day in funny notation.
    \hfill \emph{Web news reports}

    \color{darkcyan}
    14 March 2016 is $\pi \approx 3.1416$ day in funny notation.
    \hfill \emph{Web news reports}

    \color{darkorange}
    14 March 2016 is $\pi \approx 3.1416$ day in funny notation.
    14 March 2016 is $\pi \approx 3.1416$ day in funny notation.
    14 March 2016 is $\pi \approx 3.1416$ day in funny notation.
    \linebreak
    \strut
    \hfill \emph{Web news reports}
\end{quote}

Notice that \hilitebox{\texttt{quote}} paragraphs
are \emph{not} indented, but the environment
itself \emph{is} indented on the left and right by
the value of \verb=\leftmargin= (set to
\texttt{\the\leftmargin} in this document, which
should be identical to \verb=2.5em=, where
\verb=1em= = \texttt{\dimen0 = 1em \the\dimen0}).

For debugging purposes, we also have
\verb=\leftmargini= set to
\texttt{\the\leftmargini}, and we have
\verb=\leftmarginii= set to
\texttt{\the\leftmarginii}.

This is a \hilitebox{\texttt{quotation}}
environment with one paragraph, following a fairly
short paragraph of prose (notice that the
\texttt{quotation} paragraphs \emph{are}
indented):

\begin{singlespace}
\color{darkblue}
\begin{verbatim}
\begin{quotation}
    \color{blue}
    Algebra is concerned with manipulation in
    \emph{time}, and geometry is concerned with
    \emph{space}. These are two orthogonal aspects
    of the world, and they represent two different
    points of view in mathematics.  Thus the
    argument or dialogue between mathematicians in
    the past about the relative importance of
    geometry and algebra represents something very
    fundamental.
    \hfill
    \emph{Sir Michael Atiyah}
    % Mathematics in the 20$^{th}$ century
    % NTM {\bf 10}(1--3) 25--39 (September 2002)
    % http://dx.doi.org/10.1007/BF03033096
\end{quotation}
\end{verbatim}
\end{singlespace}
%
\begin{quotation}
    \color{blue}
    Algebra is concerned with manipulation in
    \emph{time}, and geometry is concerned with
    \emph{space}. These are two orthogonal aspects
    of the world, and they represent two different
    points of view in mathematics.  Thus the
    argument or dialogue between mathematicians in
    the past about the relative importance of
    geometry and algebra represents something very
    fundamental.
    \hfill
    \emph{Sir Michael Atiyah}
    % Mathematics in the 20$^{th}$ century
    % NTM {\bf 10}(1--3) 25--39 (September 2002)
    % http://dx.doi.org/10.1007/BF03033096
\end{quotation}

This is a \hilitebox{\texttt{quotation}} environment with three
paragraphs, following a fairly short paragraph of prose:

\begin{quotation}
    \color{utahred}
    Algebra is concerned with manipulation in
    \emph{time}, and geometry is concerned with
    \emph{space}. These are two orthogonal aspects
    of the world, and they represent two different
    points of view in mathematics.  Thus the
    argument or dialogue between mathematicians in
    the past about the relative importance of
    geometry and algebra represents something very
    fundamental.
    \hfill
    \emph{Sir Michael Atiyah}

    Algebra is concerned with manipulation in
    \emph{time}, and geometry is concerned with
    \emph{space}. These are two orthogonal aspects
    of the world, and they represent two different
    points of view in mathematics.  Thus the
    argument or dialogue between mathematicians in
    the past about the relative importance of
    geometry and algebra represents something very
    fundamental.
    \hfill
    \emph{Sir Michael Atiyah}

    Algebra is concerned with manipulation in
    \emph{time}, and geometry is concerned with
    \emph{space}. These are two orthogonal aspects
    of the world, and they represent two different
    points of view in mathematics.  Thus the
    argument or dialogue between mathematicians in
    the past about the relative importance of
    geometry and algebra represents something very
    fundamental.
    \hfill
    \emph{Sir Michael Atiyah}
\end{quotation}

%% \showthe \baselineskip

Now all following text should be back in
double-spaced mode, and just go on and on and on
and on and on and on and on and on and on and on
and on and on and on and on and on and on and on
and on and on and on and on and on and on and on
and on and on and on and on and on and on and on
and on and on and on and on and on and on and on
and on and on and on and on and on and on and on
and on and on and on and on and on and on and on
and on and on and on and on and on and on and on
and on and on and on and on and on and on and on
and on and on and on and on and on and on and on
and on and on and on and on \ldots{}.

%% \showthe \baselineskip

Now all following text should be back in
double-spaced mode, and just go on and on and on
and on and on and on and on and on and on and on
and on and on and on and on and on and on and on
and on and on and on and on and on and on and on
and on and on and on and on and on and on and on
and on and on and on and on and on and on and on
and on and on and on and on and on and on and on
and on and on and on and on and on and on and on
and on and on and on and on and on and on and on
and on and on and on and on and on and on and on
and on and on and on and on and on and on and on
and on and on and on and on \ldots{}.

%%% \showthe \baselineskip
