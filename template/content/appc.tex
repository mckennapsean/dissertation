%%% -*-LaTeX-*-

\chapter{The Third}

This is an appendix.

There are several books
\cite{%
    Singh:1997:FEE,%
    Salomon:1995:AT,%
    Robbins:2005:CSS,%
    Randell:1982:ODC,%
    Olver:2010:NHM,%
    Mittelbach:2004:LC,%
    Lamport:1985:LDP,%
    Knuth:1999:DT,%
    Knuth:1986:TB,%
    Knuth:1986:MB,%
    Farmelo:2009:SMH%
}
listed in our bibliography.

We also reference several journal articles
\cite{%
    Wiles:1995:MEC,%
    Taylor:1995:RTP,%
    Lahiri:2002:ADA,%
    Kim:1999:AFC,%
    Johnson:1978:LDT,%
    Huskey:1980:LLC,%
    Heilbron:1969:GBA,%
    Hall:1994:PNE,%
    Goldstine:1946:ENI,%
    Einstein:1911:BMAb,%
    Einstein:1911:BMAa,%
    Einstein:1906:NBM,%
    Cody:1981:APF,%
    Beebe:1989:PCP,%
    Babbage:1910:BBA%
}
and three famous doctoral theses of later winners
\cite{%
    Einstein:1905:NBM,%
    Dirac:1926:QM,%
    Bohr:1911:SME%
}
of the Nobel Prize in Physics (1922, 1933, and
1921):

Notice that, even though those citations appeared
in \LaTeX{} \verb=\cite{...}= commands with their
\BibTeX{} citation labels in reverse alphabetical
order, thanks to the \verb=citesort= package,
their reference-list numbers have been sorted in
numerically ascending order, and then
range-reduced.

Mention should also be made of a famous Dutch
computer scientist's first publication
\cite{Dijkstra:1953:FBV}.

Font metrics are an important, albeit low-level,
aspect of typesetting. See the \emph{Adobe
Systems} manual about that company's procedures
\cite{Adobe:1990:AFM}.

The bibliography at the end of this thesis contains several examples
of documents with non-English titles, and their \BibTeX{} entries
provide title translations following the practice recommended by the
American Mathematical Society and SIAM\@.  Here is a sample entry that
shows how to do so:
%
\begin{verbatim}
@PhdThesis{Einstein:1905:NBM,
  author =       "Albert Einstein",
  title =        "{Eine Neue Bestimmung der Molek{\"u}ldimensionen}.
                 ({German}) [{A} new determination of molecular
                 dimensions]",
  type =         "Inaugural dissertation",
  school =       "Bern Wyss.",
  address =      "Bern, Switzerland",
  year =         "1905",
  bibdate =      "Fri Dec 17 10:46:57 2004",
  bibsource =    "http://www.math.utah.edu/pub/tex/bib/einstein.bib",
  note =         "Published in \cite{Einstein:1906:NBM}.",
  acknowledgement = ack-nhfb,
  language =     "German",
  advisor =      "Alfred Kleiner (24 April 1849--3 July 1916)",
  URL =          "http://en.wikipedia.org/wiki/Alfred_Kleiner",
  remark =       "Received August 19, 1905 and published February 8,
                 1906.",
  Schilpp-number = "6",
}
\end{verbatim}

The \texttt{note} field in that entry refers to another bibliography
entry that need not have been directly cited in the document text.
Such cross-references are common in \BibTeX{} files, especially for
journal articles where there may be later comments and corrigenda that
should be mentioned.  Embedded \verb=\cite{}= commands ensure that
those possibly-important other entries are always included in the
reference list when the entry is cited.  The last bibliography entry
\cite{Wiles:1995:MEC} in this thesis has a long \texttt{note} field
that tells more about what some may view as the most important paper
in mathematics in the last century.

When entries cite other entries that cite other entries that cite
other entries that \ldots{}, multiple passes of \LaTeX{} and \BibTeX{}
are needed to ensure consistency.  That is another reason why document
compilation should be guided by a \texttt{Makefile} or a batch script,
rather than expecting the user to remember just how many passes are
needed.

\BibTeX{} entries are \emph{extensible}, in that arbitrary key\slash
value pairs may be present that are not necessarily recognized by any
bibliography style files.  The \texttt{advisor},
\texttt{acknowledgement}, \texttt{bibdate}, \texttt{bibsource},
\texttt{language}, \texttt{remark}, and \texttt{Schilpp-number} fields
are examples, and may be used by other software that processes
\BibTeX{} entries, or by humans who read the entries.  \texttt{DOI}
and \texttt{URL} fields are currently recognized by only a few styles,
but that situation will likely change as publishers demand that such
important information be included in reference lists.

In \BibTeX{} \texttt{title} fields, braces protect words, such as
proper nouns and acronyms, that cannot be downcased if the selected
bibliography style would otherwise do so.  In German, all nouns are
capitalized, and the simple way to ensure their protection is to brace
the entire German text in the title, as we did in the entry above.

The world's first significant computer program may
have been that written in 1842 by Lady Augusta Ada
Lovelace (1815--1852) for the computation of Bernoulli
numbers \cite{Huskey:1980:LLC,Kim:1999:AFC}.  She
was the assistant to Charles Babbage
(1791--1871), and they are the world's first
computer programmers. The programming language
\emph{Ada} is named after her, and is defined in
the ANSI/MIL-STD-1815A Standard; its number
commemorates the year of her birth.

We do not discuss mathematical \emph{transforms}
in this dissertation, but you can find that phrase
in the index (except that this sample thesis doesn't have one!)

\blah

\blah
\blah

\blah
