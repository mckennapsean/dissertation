%%% -*-LaTeX-*-

\chapter{The First}

This is a chapter.  Remember  that there should \emph{always}
be at least of few lines of prose after each sectional heading:
failure to do so is a disservice to your readers, and also
produces incorrect vertical spacing.

\section{The first section}

\blah

In \figref{fig1}, we have a picture, and the \LaTeX{} markup
to include it looks like this:
%
\begin{verbatim}
\begin{figure}[t]
    \centerline{\includegraphics{content/fig1}}
    \caption{The first figure.}%
    \figlabel{fig1}
\end{figure}
\end{verbatim}
%
We intentionally omitted an extension on the filename, so that this
document can be processed with \verb=latex= to get an output
\verb=.dvi= file, or with \verb=pdflatex= to get an output \verb=.pdf=
file.  The first case uses the file \verb=fig1.eps=, and the second
uses \verb=fig1.pdf=.  The \verb=distill= or \verb=ps2pdf= commands
can be used to convert from \emph{Encapulated PostScript}%
files to \emph{Portable Document Format}%
files.

\begin{figure}[t]
    \centerline{\includegraphics{content/fig1}}
    \caption{The first figure.}%
    \figlabel{fig1}
\end{figure}

\blah

\subsection{The first subsection}

\blah

\subsection{The second subsection}

\blah

\subsection{The third subsection}

\blah

\subsubsection{The first subsubsection}

\blah

\subsubsection{The second subsubsection}

\blah

\paragraph{The first numbered paragraph}

\blah

\paragraph{The second numbered paragraph}

\blah

\section{The second section}

\blah

\blah

\blah

\blah

In \shortfigref{fig2}, we have another picture.

\blah

\begin{figure}[b]
    \centerline{\includegraphics{content/fig2}}
    \caption{The second figure.}%
    \figlabel{fig2}
\end{figure}

\blah

In \shorttabref{lc-greek}, we show the 24-character lowercase Greek alphabet.

\blah

\blah

\begin{table}[b]
    \caption{Lowercase Greek letters.}%
    \tablabel{lc-greek}
    \vspace{\baselineskip}
    \begin{center}
        \newcommand {\omicron}  {o}
        \begin{tabular}{l@{\hspace{5cm}}l}
            \hline
            $\alpha$                    & alpha \\
            $\beta$                     & beta \\
            $\gamma$                    & gamma \\
            $\delta$                    & delta \\
            $\epsilon$, $\varepsilon$   & epsilon \\
            $\zeta$                     & zeta \\
            $\eta$                      & eta \\
            $\theta$, $\vartheta$       & theta \\
            $\iota$                     & iota \\
            $\kappa$                    & kappa \\
            $\lambda$                   & lambda \\
            $\mu$                       & mu \\
            $\nu$                       & nu \\
            $\xi$                       & xi \\
            $\omicron$                  & omicron \\
            $\pi$                       & pi \\
            $\rho$                      & rho \\
            $\sigma$, $\varsigma$       & sigma \\
            $\tau$                      & tau \\
            $\upsilon$                  & upsilon \\
            $\phi$, $\varphi$           & phi \\
            $\chi$                      & chi \\
            $\psi$                      & psi \\
            $\omega$                    & omega \\
            \hline
        \end{tabular}
    \end{center}
\end{table}

\blah

\blah

\blah

\blah

\blah

\section{The third section}

\blah

In \tabref{uc-greek}, we show the 24-character uppercase Greek
alphabet, 13 of which are identical with Latin letters, because the
Romans borrowed several letters from the earlier Greek alphabet.
However, the letter sounds do not always carry over: notice in
particular the different names of the letter shapes \textbf{H} and
\textbf{P}.  In Modern Greek, {\boldmath $\beta$} is pronounced
\emph{veeta}; the letter pair {\boldmath $\mu\!\pi$} is used to get a
\emph{bee} sound;

\blah

\begin{table}[t]
    \caption[Uppercase Greek letters.]
            {
                Uppercase Greek letters.  Notice that several have the
                same letter shapes as Latin letters, and for those,
                \TeX{} does not define macro names.  For convenience,
                we supply our own definitions of these macros:
                \texttt{\char 92\relax Alpha},
                \texttt{\char 92\relax Beta},
                \texttt{\char 92\relax Epsilon},
                \texttt{\char 92\relax Zeta},
                \texttt{\char 92\relax Eta},
                \texttt{\char 92\relax Iota},
                \texttt{\char 92\relax Kappa},
                \texttt{\char 92\relax Mu},
                \texttt{\char 92\relax Nu},
                \texttt{\char 92\relax Omicron},
                \texttt{\char 92\relax Rho},
                \texttt{\char 92\relax Tau}, and
                \texttt{\char 92\relax Chi}.
            }%
    \tablabel{uc-greek}
    \vspace{\baselineskip}
    \begin{center}
        \newcommand {\Alpha}    {A}
        \newcommand {\Beta}     {B}
        \newcommand {\Epsilon}  {E}
        \newcommand {\Zeta}     {Z}
        \newcommand {\Eta}      {H}
        \newcommand {\Iota}     {I}
        \newcommand {\Kappa}    {K}
        \newcommand {\Mu}       {M}
        \newcommand {\Nu}       {N}
        \newcommand {\Omicron}  {O}
        \newcommand {\Rho}      {P}
        \newcommand {\Tau}      {T}
        \newcommand {\Chi}      {X}
        \begin{tabular}{l@{\hspace{5cm}}l}
            \hline
            $\Alpha$                    & Alpha \\
            $\Beta$                     & Beta \\
            $\Gamma$                    & Gamma \\
            $\Delta$                    & Delta \\
            $\Epsilon$                  & Epsilon \\
            $\Zeta$                     & Zeta \\
            $\Eta$                      & Eta \\
            $\Theta$                    & Theta \\
            $\Iota$                     & Iota \\
            $\Kappa$                    & Kappa \\
            $\Lambda$                   & Lambda \\
            $\Mu$                       & Mu \\
            $\Nu$                       & Nu \\
            $\Xi$                       & Xi \\
            $\Omicron$                  & Omicron \\
            $\Pi$                       & Pi \\
            $\Rho$                      & Rho \\
            $\Sigma$                    & Sigma \\
            $\Tau$                      & Tau \\
            $\Upsilon$                  & Upsilon \\
            $\Phi$                      & Phi \\
            $\Chi$                      & Chi \\
            $\Psi$                      & Psi \\
            $\Omega$                    & Omega \\
            \hline
        \end{tabular}
    \end{center}
\end{table}

\blah

\blah

\blah

\section{Free software packages}

%%: \newcommand {\F} [1] {\texttt{\color{fsfcolor}#1}\index{#1@\protect \fsfname{#1}}}

\newcommand {\F} [1] {\texttt{#1}}

The Free Software Foundation offers almost 300 software packages, most
easily portable to many different operating systems and CPU platforms.
They include at least these:

\begin{sloppypar}
    \F{a2ps},
    \F{acct},
    \F{acm},
    \F{adns},
    \F{alive},
    \F{anubis},
    \F{apl},
    \F{archimedes},
    \F{aris},
    \F{aspell},
    \F{auctex},
    \F{autoconf-archive},
    \F{autoconf},
    \F{autogen},
    \F{automake},
    \F{avl},
    \F{ballandpaddle},
    \F{barcode},
    \F{bash},
    \F{bayonne},
    \F{bc},
    \F{binutils},
    \F{bison},
    \F{bool},
    \F{bpel2owfn},
    \F{c-graph},
    \F{ccaudio},
    \F{ccd2cue},
    \F{ccrtp},
    \F{ccscript},
    \F{cfengine},
    \F{cflow},
    \F{cgicc},
    \F{chess},
    \F{cim},
    \F{classpath},
    \F{classpathx},
    \F{clisp},
    \F{combine},
    \F{commoncpp},
    \F{complexity},
    \F{config},
    \F{coreutils},
    \F{cpio},
    \F{cppi},
    \F{cssc},
    \F{cursynth},
    \F{dap},
    \F{datamash},
    \F{ddd},
    \F{ddrescue},
    \F{dejagnu},
    \F{denemo},
    \F{dico},
    \F{diction},
    \F{diffutils},
    \F{dionysus},
    \F{direvent},
    \F{dismal},
    \F{dominion},
    \F{easejs},
    \F{ed},
    \F{edma},
    \F{electric},
    \F{emacs},
    \F{emms},
    \F{enscript},
    \F{fdisk},
    \F{ferret},
    \F{findutils},
    \F{fisicalab},
    \F{flex},
    \F{fontutils},
    \F{freedink},
    \F{freefont},
    \F{freeipmi},
    \F{gama},
    \F{garpd},
    \F{gawk},
    \F{gcal},
    \F{gcc},
    \F{gcide},
    \F{gcl},
    \F{gcompris},
    \F{gdb},
    \F{gdbm},
    \F{gengen},
    \F{gengetopt},
    \F{gettext},
    \F{gforth},
    \F{ggradebook},
    \F{ghostscript},
    \F{gift},
    \F{gleem},
    \F{glibc},
    \F{global},
    \F{glpk},
    \F{gmp},
    \F{gnash},
    \F{gnats},
    \F{gnatsweb},
    \F{gnu-c-manual},
    \F{gnu-crypto},
    \F{gnu-pw-mgr},
    \F{gnubatch},
    \F{gnubik},
    \F{gnucap},
    \F{gnucobol},
    \F{gnudos},
    \F{gnue},
    \F{gnugo},
    \F{gnuit},
    \F{gnujump},
    \F{gnukart},
    \F{gnumach},
    \F{gnun},
    \F{gnunet},
    \F{gnupod},
    \F{gnuprologjava},
    \F{gnuradio},
    \F{gnurobots},
    \F{gnuschool},
    \F{gnushogi},
    \F{gnusound},
    \F{gnuspeech},
    \F{gnuspool},
    \F{gnustep},
    \F{gnutls},
    \F{gnutrition},
    \F{gnuzilla},
    \F{goptical},
    \F{gperf},
    \F{gprolog},
    \F{greg},
    \F{grep},
    \F{groff},
    \F{grub},
    \F{gsasl},
    \F{gsegrafix},
    \F{gsl},
    \F{gslip},
    \F{gsrc},
    \F{gss},
    \F{gtypist},
    \F{guile-gnome},
    \F{guile-gtk},
    \F{guile-ncurses},
    \F{guile-opengl},
    \F{guile-rpc},
    \F{guile-sdl},
    \F{guile},
    \F{gv},
    \F{gvpe},
    \F{gxmessage},
    \F{gzip},
    \F{halifax},
    \F{health},
    \F{hello},
    \F{help2man},
    \F{hp2xx},
    \F{httptunnel},
    \F{hurd},
    \F{hyperbole},
    \F{idutils},
    \F{ignuit},
    \F{indent},
    \F{inetutils},
    \F{intlfonts},
    \F{jacal},
    \F{jel},
    \F{jwhois},
    \F{kawa},
    \F{less},
    \F{libcdio},
    \F{libextractor},
    \F{libffcall},
    \F{libiconv},
    \F{libidn},
    \F{libmatheval},
    \F{libmicrohttpd},
    \F{librejs},
    \F{libsigsegv},
    \F{libtasn1},
    \F{libtool},
    \F{libunistring},
    \F{libxmi},
    \F{lightning},
    \F{lilypond},
    \F{liquidwar6},
    \F{lsh},
    \F{m4},
    \F{macchanger},
    \F{mailman},
    \F{mailutils},
    \F{make},
    \F{marst},
    \F{maverik},
    \F{mc},
    \F{mcron},
    \F{mcsim},
    \F{mdk},
    \F{metahtml},
    \F{mifluz},
    \F{mig},
    \F{miscfiles},
    \F{mit-scheme},
    \F{moe},
    \F{motti},
    \F{mpc},
    \F{mpfr},
    \F{mpria},
    \F{mtools},
    \F{myserver},
    \F{nano},
    \F{ncurses},
    \F{nettle},
    \F{non-gnu},
    \F{ocrad},
    \F{octave},
    \F{oleo},
    \F{orgadoc},
    \F{osip},
    \F{paperclips},
    \F{parallel},
    \F{parted},
    \F{patch},
    \F{pem},
    \F{pexec},
    \F{phantom},
    \F{pies},
    \F{plotutils},
    \F{proxyknife},
    \F{pspp},
    \F{psychosynth},
    \F{pth},
    \F{pyconfigure},
    \F{radius},
    \F{rcs},
    \F{readline},
    \F{recutils},
    \F{reftex},
    \F{remotecontrol},
    \F{rottlog},
    \F{rpge},
    \F{rush},
    \F{sather},
    \F{sauce},
    \F{savannah},
    \F{scm},
    \F{screen},
    \F{sed},
    \F{serveez},
    \F{sharutils},
    \F{shishi},
    \F{shmm},
    \F{shtool},
    \F{sipwitch},
    \F{slib},
    \F{smalltalk},
    \F{solfege},
    \F{spacechart},
    \F{spell},
    \F{sqltutor},
    \F{src-highlite},
    \F{stow},
    \F{superopt},
    \F{swbis},
    \F{tar},
    \F{termcap},
    \F{termutils},
    \F{teseq},
    \F{teximpatient},
    \F{texinfo},
    \F{thales},
    \F{time},
    \F{tramp},
    \F{trueprint},
    \F{unifont},
    \F{units},
    \F{unrtf},
    \F{userv},
    \F{uucp},
    \F{vc-dwim},
    \F{vcdimager},
    \F{vera},
    \F{wb},
    \F{wdiff},
    \F{websocket4j},
    \F{wget},
    \F{which},
    \F{windows},
    \F{xaos},
    \F{xboard},
    \F{xhippo},
    \F{xlogmaster},
    \F{xnee},
    \F{xorriso}, and
    \F{zile}.
\end{sloppypar}

\blah

In \shortfigref{fig3}, we have yet another picture.

\blah

\blah

\blah

\blah

\begin{figure}[b]
    \centerline{\includegraphics{content/fig3}}
    \caption[The third figure.]
            {%
                The third figure.  This one has both short and long captions.
                \blah \par
                \blah \par
                \blah
            }%
    \figlabel{fig3}
\end{figure}

\blah

\blah

\blah

\blah

\blah

\blah

\blah

\blah


\section{Resizing figures}

In \shortfigref{fig4} through \figref{fig8}, we show how graphics files
can be rescaled to convenient sizes, with input like this:

\begin{figure}[p]
    \centerline{\includegraphics[scale = 0.5]{content/fig1}}
    \caption{The fourth figure (at 50\% scale).}%
    \figlabel{fig4}
\end{figure}

\begin{figure}[p]
    \centerline{\includegraphics[scale = 0.75]{content/fig1}}
    \caption{The fifth figure (at 75\% scale).}%
    \figlabel{fig5}
\end{figure}

\begin{figure}[p]
    \centerline{\includegraphics{content/fig1}}
    \caption{The sixth figure (at native size).}%
    \figlabel{fig6}
\end{figure}

\begin{figure}[p]
    \centerline{\includegraphics[scale = 1.25]{content/fig1}}
    \caption{The seventh figure (at 125\% scale).}%
    \figlabel{fig7}
\end{figure}

\begin{figure}[p]
    \centerline{\includegraphics[scale = 1.75]{content/fig1}}
    \caption{The eighth figure (at 175\% scale).}%
    \figlabel{fig8}
\end{figure}

\begin{verbatim}
\begin{figure}[p]
    \centerline{\includegraphics[scale = 0.5]{content/fig1}}
    \caption{The fourth figure (at 50\% scale).}%
    \figlabel{fig4}
\end{figure}

\begin{figure}[p]
    \centerline{\includegraphics[scale = 0.75]{content/fig1}}
    \caption{The fifth figure (at 75\% scale).}%
    \figlabel{fig5}
\end{figure}

\begin{figure}[p]
    \centerline{\includegraphics{content/fig1}}
    \caption{The sixth figure (at native size).}%
    \figlabel{fig6}
\end{figure}

\begin{figure}[p]
    \centerline{\includegraphics[scale = 1.25]{content/fig1}}
    \caption{The seventh figure (at 125\% scale).}%
    \figlabel{fig7}
\end{figure}

\begin{figure}[p]
    \centerline{\includegraphics[scale = 1.75]{content/fig1}}
    \caption{The eighth figure (at 175\% scale).}%
    \figlabel{fig8}
\end{figure}
\end{verbatim}

You can include multiple images, each with its own caption inside a
single \emph{unbreakable} \texttt{figure} environment, like this
example shown in \shortfigref{fig9} and \figref{fig10}, although
you might want to adjust interfigure vertical space with a
\verb=\vspace{}= command:

\begin{verbatim}
\begin{figure}[t]
    \centerline{\includegraphics[scale = 0.5]{content/fig1}}
    \caption{The fourth figure (at 50\% scale).}%
    \figlabel{fig9}
    \vspace{3ex}
    \centerline{\includegraphics[scale = 0.75]{content/fig1}}
    \caption{The fifth figure (at 75\% scale).}%
    \figlabel{fig10}
\end{figure}
\end{verbatim}

\begin{figure}[t]
    \centerline{\includegraphics[scale = 0.5]{content/fig1}}
    \caption[The ninth figure (at 50\% scale)]%
            {The ninth figure (at 50\% scale), boxed with the tenth figure.}
    \figlabel{fig9}
    \vspace{3ex}
    \centerline{\includegraphics[scale = 0.75]{content/fig1}}
    \caption[The tenth figure (at 75\% scale)]%
            {The tenth figure (at 75\% scale), boxed with the ninth figure.}%
    \figlabel{fig10}
\end{figure}

\blah

\blah

\blah

As a final example in this chapter, \figref{picture-mode}
shows how you can use \LaTeX{} \texttt{picture} mode for
annotating and positioning graphics images prepared outside
\LaTeX{}.  The input that produced that figure looks like
this:

%
\begin{verbatim}
\begin{figure}[t]
    %% The original image is 216bp wide by 72bp high, but we
    %% rescale it to 150 picture units divided by \unitlength:
    %% 150 / 0.75 = 112.5 mm
    \newcommand {\myfig} {\includegraphics[width = 112.5mm]{fig1}}

    \begin{center}
        %% The \unitlength is chosen to make the complete picture fit
        %% within the page margins

        \setlength{\unitlength}{0.75mm}

        %%%     insert (width,height)(lower-left-x,lower-left-y)
        \begin{picture}(170,70)(10,10)
            %% Place the included image FIRST!
            \put(10,10) {\myfig}

            %% Everything that follows OVERLAYS the original image!

            \graphpaper[10](0,0)(170,70)

            %% Mark the image center and corners by centered bullets
            \newcommand {\thedot} {\makebox (0,0) {$\bullet$}}
            \put( 85, 35) {\thedot}
            \put( 10, 10) {\thedot}
            \put( 10, 60) {\thedot}
            \put(160, 10) {\thedot}
            \put(160, 60) {\thedot}

            \put( 10, 10) {\makebox (0,0) [r] {lower-left}}
            \put(160, 10) {\makebox (0,0) [l] {lower-right}}
            \put( 10, 60) {\makebox (0,0) [r] {upper-left}}
            \put(160, 60) {\makebox (0,0) [l] {upper-right}}
        \end{picture}
    \end{center}

    \vspace{2\baselineskip}

    \caption[Using \LaTeX{} \texttt{picture} mode]
            {Using \LaTeX{} \texttt{picture} mode for figure labeling
             and positioning.}
    \figlabel{picture-mode}
\end{figure}
\end{verbatim}

\begin{figure}[t]
    %% The original image is 216bp wide by 72bp high, but we
    %% rescale it to 150 picture units divided by \unitlength:
    %% 150 / 0.75 = 112.5 mm
    \newcommand {\myfig} {\includegraphics[width = 112.5mm]{content/fig1}}

    \begin{center}
        %% The \unitlength is chosen to make the complete picture fit
        %% within the page margins

        \setlength{\unitlength}{0.75mm}

        %%%     insert (width,height)(lower-left-x,lower-left-y)
        \begin{picture}(170,70)(10,10)
            %% Place the included image FIRST!
            \put(10,10) {\myfig}

            %% Everything that follows OVERLAYS the original image!

            \graphpaper[10](0,0)(170,70)

            %% Mark the image center and corners by centered bullets
            \newcommand {\thedot} {\makebox (0,0) {$\bullet$}}
            \put( 85, 35) {\thedot}
            \put( 10, 10) {\thedot}
            \put( 10, 60) {\thedot}
            \put(160, 10) {\thedot}
            \put(160, 60) {\thedot}

            \put( 10, 10) {\makebox (0,0) [r] {lower-left}}
            \put(160, 10) {\makebox (0,0) [l] {lower-right}}
            \put( 10, 60) {\makebox (0,0) [r] {upper-left}}
            \put(160, 60) {\makebox (0,0) [l] {upper-right}}
        \end{picture}
    \end{center}

    \vspace{2\baselineskip}

    \caption[Using \LaTeX{} \texttt{picture} mode]
            {Using \LaTeX{} \texttt{picture} mode for figure labeling
             and positioning.}
    \figlabel{picture-mode}
\end{figure}

%
\section{Summary and conclusions}

\blah

\blah
\blah

\blah

%%% Index phrases should be attached to an important word of a phrase,
%%% and are usually best kept on a separate line by terminating the
%%% previous line with a percent comment without intervening space, as
%%% in this example:
%%%
%%%     \newcommand {\X} [1] {#1\index{#1}}
%%%
%%%     African ungulates,%
%%%     \index{African ungulate}
%%%     like the \X{gnu}, \X{impala}, \X{kudu}, and \X{springbok}
%%%     live mostly in hot climate and consume vegetation.
%%%
%%% However, for this document, we only want lots of index entries to
%%% populate a sample topic index.

%%% ====================================================================
%%% Cross-references for index entries should be specified only once:
