% rotate table
\begin{landscape}


% reduce paragraph spacing (for cells)
\begin{spacing}{1.0}


% alternate row colors (NOTE: if colors are weird, swap order, 2 to a 1)
\definecolor{light-gray}{gray}{0.90}
\rowcolors{2}{white}{light-gray}
\setlength\arrayrulewidth{1.0pt}

% methods table generated by Excel2LaTeX
\begin{longtable}{rl|rl|rl|rl|rl|c|p{10cm}}
    \caption{
      \label{tbl:100methods}
      100 exemplar design methods.
    } \\
    
    \rowcolor{gray!50}
     &  & \multicolumn{2}{c|}{\textit{\textbf{u}}} & \multicolumn{2}{c|}{\textit{\textbf{i}}} & \multicolumn{2}{c|}{\textit{\textbf{m}}} & \multicolumn{2}{c|}{\textit{\textbf{d}}} &  &  \\
    \rowcolor{gray!50}
    \multirow{-2}{*}{\textbf{\#}} & \multirow{-2}{*}{\textbf{method}} & \textit{\textbf{g}} & \textit{\textbf{e}} & \textit{\textbf{g}} & \textit{\textbf{e}} & \textit{\textbf{g}} & \textit{\textbf{e}} & \textit{\textbf{g}} & \textit{\textbf{e}} & \multirow{-2}{*}{\textit{\textbf{v}}} & \multirow{-2}{*}{\textbf{definition}} \\
    \hline
    
    \endfirsthead
    
    \rowcolor{white}
    \multicolumn{12}{l}
    {\textbf{\tablename\ \thetable{} -- continued from previous page}} \\ 
    \multicolumn{12}{l}{} \\
    \rowcolor{gray!50}
     &  & \multicolumn{2}{c|}{\textit{\textbf{u}}} & \multicolumn{2}{c|}{\textit{\textbf{i}}} & \multicolumn{2}{c|}{\textit{\textbf{m}}} & \multicolumn{2}{c|}{\textit{\textbf{d}}} &  &  \\
    \rowcolor{gray!50}
    \multirow{-2}{*}{\textbf{\#}} & \multirow{-2}{*}{\textbf{method}} & \textit{\textbf{g}} & \textit{\textbf{e}} & \textit{\textbf{g}} & \textit{\textbf{e}} & \textit{\textbf{g}} & \textit{\textbf{e}} & \textit{\textbf{g}} & \textit{\textbf{e}} & \multirow{-2}{*}{\textit{\textbf{v}}} & \multirow{-2}{*}{\textbf{definition}} \\
    \hline
    
    \endhead
    
    \hline
    \rowcolor{white}
    \multicolumn{12}{r}{{ \textit{... continued on the next page}}} \\
    
    \endfoot
    \endlastfoot
    
    1 & A/B testing &       &       &       &       &       & \sbt     &       & \sbt     & \sbt     & ``compare two versions of the same design to see which one performs statistically better against a predetermined goal'' \cite{Martin2012} \\
    2 & activity map & \sbt     & \sbt     &       &       &       &       &       &       &       & ``structuring activities of stakeholders and showing how they relate to one another\ldots. take a list of activities gathered during research and see how they are grouped based on their relationships'' \cite{Kumar2012} \\
    3 & AEIOU framework & \sbt     & \sbt     &       &       &       &       &       &       &       & ``organizational framework reminding the researcher to attend to, document, and code information under a guiding taxonomy of Activities, Environments, Interactions, Objects, and Users'' \cite{Martin2012} \\
    4 & affinity diagramming &       & \sbt     &       & \sbt     &       & \sbt     &       &       &       & ``process used to externalize and meaningfully cluster observations and insights from research, keeping design teams grounded in data as they design'' \cite{Martin2012} \\
    5 & algorithmic performance & \sbt     & \sbt     &       &       &       & \sbt     &       & \sbt     & \sbt     & ``quantitatively study the performance or quality of visualization algorithms\ldots. common examples include measurements of rendering speed or memory performance'' \cite{Isenberg2013a} \\
    6 & analogical reasoning & \sbt     &       & \sbt     &       &       &       &       &       & \sbt     & ``cognitive strategy in which previous knowledge is accessed and transferred to fit the current requirements of a novel situation'' \cite{Goncalves2014} \\
    7 & appearance modeling &       &       & \sbt     &       & \sbt     &       & \sbt     &       &       & ``refined model of a new idea that emphasizes visual styling'' \cite{Review2014} \\
    8 & artifact analysis & \sbt     & \sbt     &       &       &       &       &       &       & \sbt     & ``systematic examination of the material, aesthetic, and interactive qualities of objects contributes to an understanding of their physical, social, and cultural contexts'' \cite{Martin2012} \\
    9 & automated logging & \sbt     & \sbt     &       &       &       & \sbt     &       & \sbt     & \sbt     & ``captures the users' patterns of activity. simple reports - such as on the frequency of each error message, menu-item selection, dialog-box appearance, help invocation, form-field usage, or web-page access\ldots. can also capture performance data for alternative designs'' \cite{Shneiderman2004} \\
    10 & behavioral prototype &       &       &       & \sbt     &       &       &       &       &       & ``simulating situations of user activity to understand user behaviors and build early concepts\ldots. through observation and conversation, user behaviors help the team further build on the concepts'' \cite{Kumar2012} \\
    11 & beta releases &       &       &       &       &       &       & \sbt     &       & \sbt     & ``before software is released, it is sometimes given \ldots to a larger set of representative users. these users report problems with the product \ldots. often uncontrolled'' \cite{Abran2001} \\
    12 & bull's-eye diagramming & \sbt     & \sbt     &       & \sbt     &       &       &       &       &       & ``ranking items in order of importance using a target diagram\ldots. gather a set of data (e.g., issues, features, etc.)\ldots. plot the data on the target, and set priorities'' \cite{Review2014} \\
    13 & buy a feature & \sbt     & \sbt     &       & \sbt     &       & \sbt     &       &       &       & ``game in which people use artificial money to express trade-off decisions\ldots. ask [participants] to purchase features within the budget\ldots. encourage them to articulate their deliberations'' \cite{Review2014} \\
    14 & card sorting & \sbt     & \sbt     &       & \sbt     &       &       &       &       & \sbt     & ``participatory design technique that you can use to explore how participants group items into categories and relate concepts to one another'' \cite{Martin2012} \\
    15 & case study & \sbt     & \sbt     &       & \sbt     &       & \sbt     &       & \sbt     & \sbt     & ``research strategy involving in-depth investigation of single events or instances in context, using multiple sources of research evidence'' \& ``focuses on gaining detailed, intensive knowledge about a single instance or a set of related instances'' \cite{Martin2012} \\
    16 & coding & \sbt     & \sbt     &       & \sbt     &       & \sbt     &       &       & \sbt     & ``break data apart and identify concepts to stand for the data [open coding], [but] also have to put it back together again by relating those concepts [axial coding]'' \cite{Strauss1990} \\
    17 & cognitive map & \sbt     & \sbt     &       &       &       &       &       &       &       & ``reveal how people think about a problem space, and visualize how they process and make sense of their experience\ldots. most effective when used to structure complex problems and to inform decision making'' \cite{Martin2012} \\
    18 & cognitive task analysis & \sbt     & \sbt     &       &       &       &       &       & \sbt     & \sbt     & ``study of cognition in real-world contexts and professional practice at work'' \cite{Crandall2006} \\
    19 & cognitive walkthrough & \sbt     &       &       &       &       & \sbt     &       & \sbt     & \sbt     & ``usability inspection method that evaluates a system’s relative ease-of-use in situations where preparatory instruction, coaching, or training of the system is unlikely to occur'' \cite{Martin2012} \\
    20 & collage & \sbt     &       &       &       &       &       &       &       &       & ``allows participants to visually express their thoughts, feelings, desires, and other aspects of their life that are difficult to articulate using traditional means'' \cite{Martin2012} \\
    21 & competitive testing & \sbt     & \sbt     &       &       &       &       &       & \sbt     & \sbt     & ``process of conducting research to evaluate the usability and learnability of your competitors’ products\ldots. focuses on end-user behavior as they attempt to accomplish tasks'' \cite{Martin2012} \\
    22 & concept map &       &       &       & \sbt     &       &       &       &       &       & ``visual framework that allows designers to absorb new concepts into an existing understanding of a domain so that new meaning can be made'' \& ``sense-making tool that connects a large number of ideas, objects, and events as they relate to a certain domain'' \cite{Martin2012} \\
    23 & concept sketching &       &       & \sbt     &       &       &       &       &       & \sbt     & ``convert ideas into concrete forms that are easier to understand, discuss, evaluate, and communicate than abstract ideas that are described in words'' \& ``about making abstract ideas concrete'' \cite{Kumar2012} \\
    24 & concept sorting &       &       &       & \sbt     &       &       &       &       &       & ``disciplined effort to go through a collection of concepts, rationally organize them, and categorize them into groups'' \cite{Kumar2012} \\
    25 & consistency inspection & \sbt     &       &       &       &       & \sbt     &       & \sbt     & \sbt     & ``verify consistency across a family of interfaces, checking for consistency of terminology, color, layout, input and output formats, and so on'' \cite{Shneiderman2004} \\
    26 & constraint removal & \sbt     &       & \sbt     &       &       &       &       &       & \sbt     & ``barriers [are] transformed into a positive resource through which to create new ideas'' \cite{Goodwin2013a} \\
    27 & contextual inquiry & \sbt     & \sbt     &       &       &       &       &       &       & \sbt     & ``go where the customer works, observe the customer as he or she works, and talk to the customer about the work'' \cite{Beyer1997} \\
    28 & controlled experiment & \sbt     & \sbt     &       &       &       & \sbt     &       & \sbt     & \sbt     & ``help us to answer questions and identify casual relationships'' \cite{Lazar2010} \& ``widely used approach to evaluating interfaces and styles of interaction, and to understanding cognition in the context of interactions with systems\ldots. question they most commonly answer can be framed as: does making a change to the value of variable X have a significant effect on the value of variable Y?'' \cite{Cairns2008} \\
    29 & creative matrix &       &       & \sbt     &       &       &       &       &       &       & ``format for sparking new ideas at the intersections of distinct categories\ldots. ideate at intersections of the grid\ldots. encourage the teams to fill every cell of the grid'' \cite{Review2014} \\
    30 & creative toolkits & \sbt     & \sbt     & \sbt     & \sbt     & \sbt     & \sbt     &       &       &       & ``collections of physical elements conveniently organized for participatory modeling, visualization, or creative play by users, to inform and inspire design and business teams'' \& ``foster innovation through creativity'' \cite{Martin2012} \\
    31 & debugging &       &       &       &       &       &       & \sbt     &       & \sbt     & ``activity to find and fix bugs (faults) in the source code (or design) \ldots. purpose of debugging is to find out why a program doesn't work or produces a wrong result or output'' \cite{Abran2001} \\
    32 & diagramming &       &       &       &       & \sbt     &       &       &       &       & ``can effectively clarify structural relationships, describe processes, show how value flows through the system, show how the system evolves over time, map interactions between components, or work with other similar aspects of the system'' \& ``process of translating your ideas into diagrams helps reduce ambiguity'' \cite{Kumar2012} \\
    33 & documentation &       &       &       &       &       &       & \sbt     &       & \sbt     & ``online help, manuals, and tutorials \ldots to provide training, reference, and reminders about specific features and syntax'' \cite{Shneiderman2004} \& ``document relevant facts, significant risks and tradeoffs, and warnings of undesirable or dangerous consequences from use or misuse of software'' \& ``for external stakeholders \ldots provide information needed to determine if the software is likely to meet the \ldots users' needs'' \cite{Abran2001} \\
    34 & ergonomics evaluation & \sbt     & \sbt     &       &       &       & \sbt     &       & \sbt     & \sbt     & ``assessment of tools, equipment, devices, workstations, workplaces, or environments, to optimize the fit, safety, and comfort of use by people'' \& ``five criteria: size, strength, reach, clearance, \& posture'' \cite{Martin2012} \\
    35 & example exposure &       &       & \sbt     &       & \sbt     &       &       &       & \sbt     & ``excite ideas by exposing the subject to a solution for the same problem'' \cite{Hernandez2010} \\
    36 & excursion & \sbt     &       & \sbt     &       &       &       &       &       & \sbt     & ``participants remove themselves from a task, take a mental or physical journey to seek images or stimuli and then bring these back to make connections with the task'' \cite{Goodwin2013a} \\
    37 & experience prototyping &       &       &       & \sbt     & \sbt     & \sbt     &       &       &       & ``fosters active participation to encounter a live experience with products, systems, services, or spaces'' \cite{Martin2012} \\
    38 & field notes (diary, journal) & \sbt     & \sbt     &       &       &       &       &       &       &       & ``four types of field notes: jottings, the diary, the log, and the notes'' \& ``keep a note pad with you at all times and make field jottings on the spot'' \& ``a diary chronicles how you feel and how you perceive your relations with others around you'' \& ``a log is a running account of how you plan to spend your time, how you actually spend your time, and how much money you spent'' \& ``three kinds of notes: notes on method and technique; ethnographic, or descriptive notes; and the notes that discuss issues or provide an analysis of social situations'' \cite{Bernard2011} \\
    39 & five W's & \sbt     & \sbt     &       &       &       &       &       &       & \sbt     & ``popular concept for information gathering in journalistic reporting \ldots. captures all aspects of a story or incidence: who, when, what, where, and why'' \cite{Zhang2013a,Schulz2013a} \\
    40 & focus group & \sbt     & \sbt     &       &       &       & \sbt     &       & \sbt     & \sbt     & ``small group of well-chosen people\ldots guided by a skilled moderator\ldots [to] provide deep insight into themes, patterns, and trends'' \cite{Martin2012} \\
    41 & foresight scenario &       &       & \sbt     &       & \sbt     &       &       &       &       & ``considering hypothetical futures based on emergent trends and then formulating alternative solutions designed to meet those possible situations'' \cite{Kumar2012} \\
    42 & frame of reference shifting &       &       & \sbt     &       &       &       &       &       &       & ``change how objectives and requirements are being viewed, perceived, and interpreted'' \cite{Hernandez2010} \\
    43 & grafitti walls & \sbt     & \sbt     & \sbt     &       &       & \sbt     &       & \sbt     &       & ``open canvas on which participants can freely offer their written or visual comments about an environment or system, directly in the context of use'' \cite{Martin2012} \\
    44 & heuristic evaluation &       &       &       &       &       & \sbt     &       & \sbt     & \sbt     & ``informal usability inspection method that asks evaluators to assess an interface against a set of agreed-upon best practices, or usability 'rules of thumb''' \cite{Martin2012} \\
    45 & idea evaluation &       &       &       & \sbt     &       &       &       &       &       & ``evaluating ideas with regard to four dimensions - novelty, workability, relevance, and specificity'' \& ``novelty: nobody has expressed it before'' \& ``workability: does not violate known constraints or \ldots easily implemented'' \& ``relevance: satisfies the goals set by the problem solver'' \& ``specificity: worked out in detail'' \cite{Dean2006} \\
    46 & ideation game &       &       & \sbt     &       & \sbt     &       &       &       &       & ``engaging stakeholders in game-like activities to generate concepts'' \& ``game-building and game-playing mindsets allow participants to cut through barriers of creativity and think more openly'' \cite{Kumar2012} \\
    47 & image quality analysis & \sbt     & \sbt     &       &       &       & \sbt     &       & \sbt     & \sbt     & ``classical form of qualititative result inspection\ldots the qualitative discussion of images produced by a (rendering) algorithm\ldots. common to show and assess visually that quality goals had been met'' \cite{Isenberg2013a} \\
    48 & importance/difficulty matrix &       & \sbt     &       & \sbt     &       & \sbt     &       &       &       & ``a quad chart for plotting items by relative importance and difficulty \ldots make a poster showing a large quad chart, label horizontal axis Importance, label vertical axis Difficulty \ldots plot items horizontally by relative importance, plot items vertically by relative difficulty \ldots look for related groupings, and set priorities'' \cite{Review2014} \\
    49 & incubation &       &       & \sbt     &       &       &       &       &       &       & ``add programmed delay to allow sub-conscious processing to take place'' \cite{Hernandez2010} \\
    50 & interactive tutorial &       &       &       &       &       &       & \sbt     &       & \sbt     & ``uses the electronic medium to teach the novice user by showing simulations of the working system, by displaying attractive animations, and by engaging the user in interactive sessions''\cite{Shneiderman2004} \& ``[present] the work-product to the other participants \ldots. [take] the role of explaining and showing the material to participants'' \cite{Abran2001} \\
    51 & interviewing & \sbt     & \sbt     &       & \sbt     &       & \sbt     &       & \sbt     & \sbt     & ``fundamental research method for direct contact with participants, to collect firsthand personal accounts of experience, opinions, attitudes, and perceptions'' \& unstructured vs. guided vs. structured \cite{Martin2012} \\
    52 & key performance indicators &       &       &       &       &       & \sbt     &       & \sbt     &       & ``critical success factors for your product or service'' \& ``quantifiable, widely accepted business goals'' \& ``reflect the activities of real people'' \cite{Martin2012} \\
    53 & literature review & \sbt     & \sbt     &       &       &       &       &       &       & \sbt     & ``distill information from published sources, capturing the essence of previous research or projects as they might inform the current project'' \& ``collect and synthesize research on a given topic'' \cite{Martin2012} \\
    54 & love/breakup letters & \sbt     & \sbt     &       & \sbt     &       & \sbt     &       & \sbt     &       & ``personal letter written to a product\ldots [to reveal] profound insights about what people value and expect from the objects in their everyday lives'' \cite{Martin2012} \\
    55 & measuring users (eye tracking) & \sbt     & \sbt     &       &       &       & \sbt     &       & \sbt     & \sbt     & ``understanding what people do, how they do it, and how they react\ldots. participants in research studies can be important data sources\ldots. eye-tracking tools that tell us where people are looking on a screen\ldots. skin response or cardiovascular monitors can provide insight into a user's level of arousal or frustration'' \cite{Lazar2010} \\
    56 & mindmapping &       &       & \sbt     &       &       &       &       &       &       & ``visual thinking tool that can help generate ideas and develop concepts when the relationships among many pieces of related information are unclear'' \& also: graphic organizer, brainstorming web, tree diagram, flow diagram \cite{Martin2012} \\
    57 & morphological synthesis &       &       & \sbt     &       &       &       &       &       &       & ``organizing concepts under user-centered categories and combining concepts to form solutions\ldots a solution is a set of concepts that work together as a complete system'' \cite{Kumar2012} \\
    58 & observation & \sbt     & \sbt     &       & \sbt     &       & \sbt     &       & \sbt     & \sbt     & ``attentive looking and systematic recording of phenomena: including people, artifacts, environments, events, behaviors and interactions'' \cite{Martin2012} \& e.g. participant vs. fly-on-the-wall, axis from obtrusive to unobtrusive like in the field of ethnography \cite{Lazar2010} \\
    59 & online forum &       &       &       &       &       &       & \sbt     & \sbt     &       & ``permit posting of open messages and questions'' \& also known as: mailing lists, bulletin boards, etc. \cite{Shneiderman2004} \\
    60 & online suggestions &       &       &       &       &       &       &       & \sbt     &       & ``allow users to send messages to the maintainers or designers\ldots. encourages some users to make productive comments'' \cite{Shneiderman2004} \\
    61 & paper prototyping &       &       & \sbt     &       & \sbt     &       &       &       & \sbt     & ``create a paper-based simulation of an interface to test interaction with a user'' \cite{Maguire2001} \\
    62 & parallel prototyping &       &       & \sbt     &       & \sbt     &       &       &       & \sbt     & ``creating multiple alternatives in parallel may encourage people to more effectively discover unseen constraints and opportunities, enumerate more diverse solutions, and obtain more authentic and diverse feedback from potential users'' \& ``[this method] yields better results, more divergent ideas, and [designers] react more positively to critique'' \cite{Dow2010} \\
    63 & personas &       & \sbt     &       &       &       &       &       &       &       & ``consolidate archetypal descriptions of user behavior patterns into representative profiles, to humanize design focus, test scenarios, and aid design communication'' \cite{Martin2012} \\
    64 & photo studies & \sbt     &       &       &       &       &       &       &       &       & ``invite the participant to photo-document aspects of his or her life and interactions, providing the designer with visual, self-reported insights into user behaviors and priorities'' \cite{Martin2012} \\
    65 & pilot testing &       &       &       &       &       &       &       & \sbt     & \sbt     & ``placing offerings in the marketplace to learn how they perform and how users experience them\ldots. method for testing innovation solutions by placing them in contexts where they function as real offerings'' \cite{Kumar2012} \\
    66 & POEMS framework & \sbt     & \sbt     &       &       &       &       &       &       &       & ``observational research framework used to make sense of the elements present in a context\ldots. five elements are: People, Objects, Environments, Messages, and Services'' \cite{Kumar2012} \\
    67 & prototyping &       &       & \sbt     &       & \sbt     &       &       &       & \sbt     & ``tangible creation of artifacts at various levels of resolution, for development and testing of ideas within design teams and with clients and users'' \cite{Martin2012} \\
    68 & provocative stimuli &       &       & \sbt     &       & \sbt     &       &       &       & \sbt     & ``trigger new ideas by exposing the subject to related and unrelated pointers, pictures, sounds'' \cite{Hernandez2010} \\
    69 & questionnaire & \sbt     & \sbt     &       & \sbt     &       & \sbt     &       & \sbt     & \sbt     & ``survey instruments designed for collecting self-report information from people about their characteristics, thoughts, feelings, perceptions, behaviors, or attitudes, typically in written form'' \cite{Martin2012} \\
    70 & reflection &       & \sbt     &       & \sbt     &       &       &       &       & \sbt     & ``[ask participants] what they knew\ldots that they hadn't known at the outset'' \cite{Goodwin2013a} \\
    71 & roadmap &       &       &       &       & \sbt     &       & \sbt     &       & \sbt     & ``plan for implementing solutions\ldots. helps explore how solutions are to be built up, with short-term initiatives as a foundation on which long-term solutions are based'' \& ``prioritizing the order of implementation'' \cite{Kumar2012} \\
    72 & role-playing & \sbt     & \sbt     & \sbt     & \sbt     & \sbt     & \sbt     & \sbt     & \sbt     &       & ``acting the role of the user in realistic scenarios can forge a deep sense of empathy and highlight challenges, presenting opportunities that can be met by design'' \cite{Martin2012} \\
    73 & rose-thorn-bud &       & \sbt     &       & \sbt     &       & \sbt     &       &       &       & ``technique for identifying things as positive, negative, or having potential'' \& tag outcomes as rose, thorn, or bud, accordingly \cite{Review2014} \\
    74 & round robin &       &       & \sbt     & \sbt     &       &       &       &       &       & ``activity in which ideas evolve as they are passed from person to person'' \cite{Review2014} \\
    75 & sample data &       &       &       &       &       &       & \sbt     &       & \sbt     & ``create benchmark datasets\ldots provide real data and tasks \ldots. illustrating [tools] with convincing examples using real data'' \cite{Plaisant2004} \\
    76 & semantic differential & \sbt     & \sbt     &       & \sbt     &       & \sbt     &       & \sbt     &       & ``linguistic tool designed to measure people’s attitudes toward a topic, event, object, or activity, so that its deeper connotative meaning can be ascertained'' \cite{Martin2012} \\
    77 & simulation & \sbt     &       &       &       &       & \sbt     &       &       &       & ``deep approximations of human or environmental conditions, designed to forge an immersive, empathic sense of real-life user experiences'' \cite{Martin2012} \\
    78 & social mapping & \sbt     & \sbt     &       &       &       &       &       &       &       & ``a visual representation of relationships between objects and spaces \ldots. maps reflect people's beliefs about the spaces and objects around them: how they define those spaces, how they categorize them, and what they feel about them'' \cite{Goodman2012} \\
    79 & spatial mapping & \sbt     & \sbt     &       &       &       &       &       &       &       & ``a visual representation of relationships between people \ldots. maps reflect people's beliefs about the spaces and objects around them: how they define those spaces, how they categorize them, and what they feel about them'' \cite{Goodman2012} \\
    80 & speed dating &       &       &       & \sbt     &       & \sbt     &       &       &       & ``compare multiple design concepts in quick succession'' \& ``exposing people to future design ideas via storyboards and simulated environments before any expensive technical prototypes are built'' \cite{Martin2012} \\
    81 & stakeholder feedback &       & \sbt     &       & \sbt     &       & \sbt     &       & \sbt     & \sbt     & ``demoing the visualization to a group of people, often and preferably domain experts, letting them ``play'' with the system and / or observe typical system features as shown by the representatives'' \cite{Lam2011a} \\
    82 & stakeholder map & \sbt     & \sbt     &       &       &       &       &       &       &       & ``visually consolidate and communicate the key constituents of a design project'' \cite{Martin2012} \\
    83 & statistical analysis & \sbt     & \sbt     &       & \sbt     &       & \sbt     &       & \sbt     & \sbt     & ``many critical decisions need to be made when analyzing data, such as the type of statistical method to be used, the confidence threshold, as well as the interpretation of the significance test results'' \cite{Lazar2010} \\
    84 & storyboarding &       &       & \sbt     &       & \sbt     &       &       &       & \sbt     & ``visually capture the important social, environmental, and technical factors that shape the context of how, where, and why people engage with products'' \& ``build empathy for end users'' \cite{Martin2012} \\
    85 & suspended judgement &       &       & \sbt     &       &       &       &       &       &       & ``postpone premature decisions or dismissing an idea'' \& ``generate as many ideas as possible'' \cite{Hernandez2010} \\
    86 & task analysis & \sbt     & \sbt     &       &       &       & \sbt     &       & \sbt     & \sbt     & ``breaks down the constituent elements of a user’s work flow, including actions and interactions, system response, and environmental context'' \& can be conducted on a tool or a human \cite{Martin2012} \\
    87 & technology probe & \sbt     & \sbt     & \sbt     & \sbt     & \sbt     & \sbt     &       &       & \sbt     & ``simple, flexible, and adaptable technologies with three interdisciplinary goals: the social science goal of understanding the needs and desires of users in a real-world setting, the engineering goal of field-testing the technology, and the design goal of inspiring users and researchers to think about new technologies'' \cite{Hutchinson2003a} \\
    88 & think-aloud protocol & \sbt     &       &       &       &       & \sbt     &       & \sbt     & \sbt     & ``asks people to articulate what they are thinking, doing, or feeling as they complete a set of tasks that align with their realistic day-to-day goals'' \cite{Martin2012} \\
    89 & thought experiment &       & \sbt     &       & \sbt     &       & \sbt     &       &       &       & ``think about research questions as if it were possible to test them in true experiments\ldots. what would the experiment look like?'' \cite{Bernard2011} \\
    90 & usability report & \sbt     & \sbt     &       &       &       & \sbt     &       & \sbt     & \sbt     & ``focuses on people and their tasks, and seeks empirical evidence about how to improve the usability of an interface'' \cite{Martin2012} \\
    91 & usability testing & \sbt     & \sbt     &       &       &       & \sbt     &       & \sbt     & \sbt     & ``carried out by observing how participants perform a set of predefined tasks\ldots. take notes of interesting observed behaviors, remarks voiced by the participant, and major problems in interaction'' \cite{Lam2011a} \\
    92 & user journey map & \sbt     & \sbt     &       &       &       &       &       &       &       & ``breaks down a users' journey into component parts to gain insights into problems that may be present or opportunities for innovations\ldots. activities are shown as nodes'' \cite{Kumar2012} \\
    93 & video ethnography & \sbt     &       &       &       &       &       &       &       &       & ``capture peoples' activities and what happens in a situation as video that can be analyzed for recognizing behavioral patterns and insights'' \& ``similar to photo ethnography'' \cite{Kumar2012} \\
    94 & video scenario &       &       &       & \sbt     &       & \sbt     &       &       &       & ``short movie showing the attributes of a new concept in use\ldots. identify a new concept to represent\ldots. record video or take still photos of each scene'' \cite{Review2014} \\
    95 & visual metrics & \sbt     & \sbt     &       &       &       & \sbt     &       & \sbt     & \sbt     & ``automatic procedures which compare one solution to another\ldots. based on the definition of one or more image quality measures that capture the effectiveness of the visual output according to a desired property of the visualization'' \cite{Lam2011a} \\
    96 & voting &       & \sbt     &       & \sbt     &       & \sbt     &       & \sbt     & \sbt     & ``a quick poll of collaborators to reveal preferences and opinions'' \cite{Review2014} \\
    97 & weighted matrix &       & \sbt     &       & \sbt     &       & \sbt     &       &       &       & ``matrix ranks potential design opportunities against key success criteria'' \& ``help identify and prioritize the most promising opportunities'' \cite{Martin2012} \\
    98 & wireframing &       &       & \sbt     &       & \sbt     &       &       &       & \sbt     & ``schematic diagramming: an outline of the structure and essential components of a system'' \cite{Review2014} \\
    99 & wishful thinking & \sbt     &       & \sbt     &       &       &       &       &       & \sbt     & ``[participants are] asked to think about aspirations for [their domain]\ldots. what would you like to know? what would you like to be able to do? whta would you like to see?'' \cite{Goodwin2013a} \\
    100 & wizard-of-oz &       &       &       &       &       & \sbt     &       &       & \sbt     & ``participants are led to believe they are interacting with a working prototype of a system, but in reality, a researcher is acting as a proxy for the system from behind the scenes'' \cite{Martin2012} \\
    
    \hline
\end{longtable}

% end page and un-rotate
\newpage
\end{spacing}
\end{landscape}
