%%% -*-LaTeX-*-
%%% ====================================================================
%%% This is a sample top-level LaTeX-2e file for typesetting a thesis
%%% or dissertation at the University of Utah.  Most students find it
%%% convenient to start with a COPY of this file as a template, and
%%% then alter that copy to match their needs.
%%%
%%% There is an associated Unix Makefile that can be similarly
%%% customized, and then the only command ever needed to typeset the
%%% complete thesis is the single word "make".  Of course, during
%%% writing and typesetting, not all of the steps are needed, so
%%% often, one can just name a convenience target such as "make
%%% dvi-pass" or "make pdf-pass" to do just a single pass of LaTeX and
%%% BibTeX.
%%%
%%% There should be no, or very few, macro definitions in this file;
%%% any needed belong in a private style file, called mythesis.sty,
%%% and input below after all other packages.  The bulk of this file
%%% should just be command invocations, and any arguments that they
%%% need.
%%%
%%% We exploit the fact that TeX ignores spaces after command names to
%%% line up arguments for better readability.
%%%
%%% Each chapter should be a separate complete file, so that you can
%%% insert a command like \includeonly{chap_intro} before the first
%%% \include{chap_xxx} command to avoid typesetting all but the
%%% chapter that you are currently working on, to save time.
%%%
%%% Remember that occupants of job positions change jobs from time to
%%% time: YOU are responsible for ensuring correct names of all humans
%%% mentioned in this file!
%%%
%%% [16-Mar-2016]
%%% ====================================================================

\documentclass[11pt,Chicago]{styles/uuthesis2e}

%%% Undefine two macros from uuthesis2e.cls that conflict with
%%% definitions in amsthm.sty that fail to check for prior definitions!
%%% NB: The amsthm package refines the LaTeX theorem environment,
%%% and the uuthesis-color-headings wraps that definition, so the
%%% amsthm package must be read first!
% \let \proof    = \relax
% \let \endproof = \relax
% \usepackage{amsthm}
% load in math packages for s-corrplot paper
\usepackage{amsfonts}
\usepackage{amsmath}

%%% ====================================================================
%%% Choose an alternate font family for the document if the TeX default
%%% of Computer Modern is not wanted:
% \usepackage{DejaVuSerifCondensed}
% \usepackage{fourier}
\usepackage{libertine}

%%% ====================================================================
%%% Some miscellaneous Utah- and student-specific settings:
%%%
%%% Chapter is one level, section and subsection are the next two levels.
\fourlevels

%%% ====================================================================
%%% The remaining packages are required by this particular thesis,
%%% but other theses will almost certainly need different packages:
%%%
%%% WARNING: MANY \LaTeX{} packages change dimensions, glue, and/or
%%% formatting styles, and such changes are likely to conflict with
%%% University of Utah Thesis Office requirements.  Therefore, minimize
%%% the number of packages that you include!
\usepackage{graphicx}
\usepackage{booktabs} % better latex tables
\usepackage{longtable} % tables beyond 1 page
% \usepackage{url} % must be careful when changing formatting for text/hypertext links!
\usepackage{afterpage} % execute command after page break
% \usepackage{varioref} % smart referencing (e.g. on this page, on page XX)
% \usepackage{subfigure} % create nested figure structures
\usepackage{multirow} % formatting cells going beyond one row

%%% ====================================================================
%%% Support for a subject index:
%% \usepackage{uuthesis-index}

%%% ====================================================================
%%% The various uuthesis-*.sty packages must come AFTER all other
%%% system-provided packages, so that they can correctly override
%%% settings from those packages.

%%% Include latest updates for 2016 (WARNING: the name is subject to
%%% change: see http://www.math.utah.edu/pub/uuthesis/ for the most
%%% current version.)
\usepackage{styles/uuthesis-2016-h}  % MANDATORY package

%%% This is an OPTIONAL package that sets chapter and sectional headings
%%% in color:
%%% Use one or the other of these:
\usepackage{color}
\usepackage{styles/uuthesis-color-headings}
%%%
%%% NB: Be careful with use of colors in typesetting, and in figures,
%%% because about 6 percent of the human male population is red/green
%%% color blind: they see those colors as shades of brown.  Red and
%%% blue, or blue and green, are better choices for choosing
%%% distinguishable colors.  Also, avoid light colors, especially
%%% yellow, because they are hard to see against white paper and
%%% screen backgrounds, and when printed on black-and-white printers,
%%% where they are rendered in gray, they may be too faint to read.

%%% ====================================================================
%%% Support for a subject index:
%: \usepackage{uuthesis-index}

%%% ====================================================================
%%% This single user-specific file is where all personal customizations
%%% and macro definitions should be placed, and it should come LAST,
%%% after ALL OTHER packages, in case it needs to override some of their
%%% definitions.
\usepackage{styles/custom}

% bibliography style guide (IEEE)
% safely load bibliography in utf-8, don't sort (sorts in-text-order)
% include abbreviation files for journals
% use double-spaces between references
\usepackage[style=ieee,bibencoding=utf8,backend=biber,sorting=none]{biblatex}
\addbibresource{styles/journal-abbreviations.bib}
\addbibresource{references.bib}
\setlength\bibitemsep{2\itemsep}


%%% ====================================================================
%%% The student-specific front matter fields are defined here:
\author                 {Sean Patrick McKenna}
\title                  {The Design Activity Framework: Investigating the Data Visualization Design Process}
\thesistype             {dissertation}

\dedication             {To ...}

%%% Most students need just a short degree name, like this:
% \degree                 {Doctor of Philosophy}
%%% However, multiline degrees are possible, and are done like this:
\degree                 {Doctor of Philosophy \\
                        in \\
                        Computing}

%%% College- and Department-level definitions:
\approvaldepartment     {Computing}
\department             {School of Computing}
\graduatedean           {David B. Kieda}
\departmentchair        {Ross Whitaker}

%%% The graduate student's committee members:
\committeechair         {Miriah Meyer}
\firstreader            {Alexander Lex}
\secondreader           {Tamara Denning}
\thirdreader            {James Agutter}
\fourthreader           {Nathalie Henry Riche}
\chairtitle             {Professor}


%%% NB: It is rare, but possible, for there to be two chairs, For
%%% example, one student had
%%%
%%% \committeechair{\mbox{\small Andrej Cherkaev and Andrejs Treibergs}}
%%%
%%% The \mbox{} ensures that line breaks cannot happen, and the \small
%%% is necessary to make the names fit on the Dissertation Approval form

%%% Dates that must be adjusted for each academic term, and be permitted
%%% according to the University of Utah Thesis Office:
\submitdate             {August 2017}
\copyrightyear          {2017}

%%% Dates on which committee members approved the thesis
\chairdateapproved      {01 June 2017}
\firstdateapproved      {01 June 2017}
\seconddateapproved     {01 June 2017}
\thirddateapproved      {01 June 2017}
\fourthdateapproved     {01 June 2017}

%%%%%
% what did these do....?
% \citeindextrue
% \makeindex
% \includeonly{chapters/intro,chapters/background,chapters/framework,chapters/formulation,chapters/security,chapters/story,chapters/timeline}

%\inputpicturetrue % By Jeff McGough. See uuguide and private thesis.sty
%\inputpicturefalse % To NOT produce pictures, uncomment this line
%%%%%


%%% ====================================================================
%%% Typesetting begins here:

\begin{document}

%% contents of dissertation
\frontmatterformat
\titlepage
\copyrightpage
\dissertationapproval
\setcounter {page}     {2}             % UofU Thesis Office demands abstract on p. iii: start one lower
\preface    {tex/00-abstract} {Abstract}
% \dedicationpage
\tableofcontents
% \listoffigures
% \listoftables

%%% Uncomment this is you want the contents of acknowledge.tex typeset here.
%%% Note that both "Acknowledgement" and "Acknowledgment" are accepted
%%% spellings of that word.
\preface{tex/10-acknowledge}{Acknowledgements}

%%% Demonstrations of thesis typesetting features for the sample thesis.
%%% Once you have seen the examples, you can comment out this line.
%%: \optionalfront {Typesetting Experiments} {%%% -*-LaTeX-*-
%%% ====================================================================
%%% This file is intended to be included as a front-matter section of
%%% the sample thesis, with a command like this
%%%
%%% \optionalfront{Typesetting Experiments}{%%% -*-LaTeX-*-
%%% ====================================================================
%%% This file is intended to be included as a front-matter section of
%%% the sample thesis, with a command like this
%%%
%%% \optionalfront{Typesetting Experiments}{%%% -*-LaTeX-*-
%%% ====================================================================
%%% This file is intended to be included as a front-matter section of
%%% the sample thesis, with a command like this
%%%
%%% \optionalfront{Typesetting Experiments}{\input{samples}}
%%%
%%% just before the \maintext macro that begins the body of the thesis,
%%% and starts chapter numbering.
%%%
%%% [16-Mar-2016]
%%% ====================================================================

\input{rgb.sty}

\definecolor{utahred} {rgb} {0.8, 0.0, 0.0} % official definition for University of Utah Printing Services

\doublespace

%% \showthe \baselineskip

In this section, we use color in several places.
The \verb=\colorbox= command takes two arguments
--- a named color and text to be in black on a
background of that color --- and sets the text in
a box with a small margin of width \verb=\fboxsep=
(set to \texttt{\the\fboxsep} in this document).

Here, we want a tighter colored box that has a
fixed height, and is independent of letter shape.
We set the margin to zero inside a group so that
the change is purely local, and so that height and
depth of the line are not increased over what they
would be if the colored box were not used.  We
prefix a \TeX{} \verb=\strut= to the user-supplied
text, because that command expands to a zero-width
box of the height and depth of parentheses, which,
in most fonts, delimit the extent of letter
shapes.

\begin{verbatim}
\newcommand {\hilitebox} [1] {{\fboxsep = 0pt\colorbox{pink}{\strut #1}}}
\end{verbatim}

\newcommand {\hilitebox} [1] {{\fboxsep = 0pt\colorbox{pink}{\strut #1}}}

Here is a fragment from the first chapter in another
thesis, set in \emph{emphasized text} to distinguish it
from the rest of this section:

\begin{itshape}
    In light of the known results, the consistency
    of empirical semivariogram and related
    estimators is widely considered a settled
    matter.  For example, Lahiri, Lee, and Cressie
    \cite{Lahiri:2002:ADA} state:

    \begin{quote}
        The simpler and more commonly used
        nonparametric estimators of the variogram,
        such as the method of moments estimator of
        Matheron (1962) and its robustified
        versions due to Cressie and Hawkins (1980)
        have many desirable properties like,
        unbiasedness, consistency, etc. \ldots
    \end{quote}
    \noindent
    Regarding a kernel estimator of the covariance
    function, Hall and Patil
    \cite{Hall:1994:PNE} remarked:
    \begin{quote}
        It is not difficult to see that if, as $ n
        $ increases, the points $ t_i $ become
        increasingly dense in each bounded subset
        of $ \mathbb{R}^d $, then the bandwidth $
        h $ may be chosen so that $ \check \rho(t)
        \to \rho(t) $ as $ n \to \infty $, for
        each $ t \in \mathbb{R}^d $.
    \end{quote}
    However, in order to be true, such statements
    would need to be qualified by many assumptions
    on the random field as well as on the
    observation locations. We will see in
    \S2.3 that even for well-behaved
    random fields (e.g., $\rho^*$-mixing Gaussian
    random fields), it is not enough to assume
    that the observation locations become
    increasingly dense in each bounded subset; a
    stronger assumption must be made to ensure
    that the observation locations do not become
    denser in one region too much faster than in
    others.
\end{itshape}

The text before the previous paragraph contained
two \texttt{quote} environments separated by a
line of prose.  Here are some more tests of both
kinds of \LaTeX{} environments for showing text
written by someone else.

This is a \hilitebox{\texttt{quote}} environment
with one short line, following a fairly short
paragraph of prose (in this, and following
examples, the text is explicitly colored with a
command like \verb=\color{purple}= inside the
environment before the text):
%
\begin{singlespace}
\color{darkblue}
\begin{verbatim}
\begin{quote}
    \color{purple}
    14 March 2016 is $\pi \approx 3.1416$ day in funny notation.
    \hfill \emph{Web news reports}
\end{quote}
\end{verbatim}
\end{singlespace}
%
\begin{quote}
    \color{purple}
    14 March 2016 is $\pi \approx 3.1416$ day in funny notation.
    \hfill \emph{Web news reports}
\end{quote}

This is a \hilitebox{\texttt{quote}} environment
with three short lines, each a separate paragraph,
following a fairly short paragraph of prose.

\begin{singlespace}
\color{darkblue}
\begin{verbatim}
\begin{quote}
    \color{forestgreen}
    14 March 2016 is $\pi \approx 3.1416$ day in funny notation.
    \hfill \emph{Web news reports}

    14 March 2016 is $\pi \approx 3.1416$ day in funny notation.
    \hfill \emph{Web news reports}

    14 March 2016 is $\pi \approx 3.1416$ day in funny notation.
    \hfill \emph{Web news reports}
\end{quote}
\end{verbatim}
\end{singlespace}
%
\begin{quote}
    \color{forestgreen}
    14 March 2016 is $\pi \approx 3.1416$ day in funny notation.
    \hfill \emph{Web news reports}

    14 March 2016 is $\pi \approx 3.1416$ day in funny notation.
    \hfill \emph{Web news reports}

    14 March 2016 is $\pi \approx 3.1416$ day in funny notation.
    \hfill \emph{Web news reports}
\end{quote}

Here is another example, this time with separate
colors for each paragraph:

\begin{singlespace}
\color{darkblue}
\begin{verbatim}
\begin{quote}
    \color{darkkhaki}
    14 March 2016 is $\pi \approx 3.1416$ day in funny notation.
    \hfill \emph{Web news reports}

    \color{darkmagenta}
    14 March 2016 is $\pi \approx 3.1416$ day in funny notation.
    \hfill \emph{Web news reports}

    \color{darkcyan}
    14 March 2016 is $\pi \approx 3.1416$ day in funny notation.
    \hfill \emph{Web news reports}

    \color{darkorange}
    14 March 2016 is $\pi \approx 3.1416$ day in funny notation.
    14 March 2016 is $\pi \approx 3.1416$ day in funny notation.
    14 March 2016 is $\pi \approx 3.1416$ day in funny notation.
    \linebreak
    \strut
    \hfill \emph{Web news reports}
\end{quote}
\end{verbatim}
\end{singlespace}
%
\begin{quote}
    \color{darkkhaki}
    14 March 2016 is $\pi \approx 3.1416$ day in funny notation.
    \hfill \emph{Web news reports}

    \color{darkmagenta}
    14 March 2016 is $\pi \approx 3.1416$ day in funny notation.
    \hfill \emph{Web news reports}

    \color{darkcyan}
    14 March 2016 is $\pi \approx 3.1416$ day in funny notation.
    \hfill \emph{Web news reports}

    \color{darkorange}
    14 March 2016 is $\pi \approx 3.1416$ day in funny notation.
    14 March 2016 is $\pi \approx 3.1416$ day in funny notation.
    14 March 2016 is $\pi \approx 3.1416$ day in funny notation.
    \linebreak
    \strut
    \hfill \emph{Web news reports}
\end{quote}

Notice that \hilitebox{\texttt{quote}} paragraphs
are \emph{not} indented, but the environment
itself \emph{is} indented on the left and right by
the value of \verb=\leftmargin= (set to
\texttt{\the\leftmargin} in this document, which
should be identical to \verb=2.5em=, where
\verb=1em= = \texttt{\dimen0 = 1em \the\dimen0}).

For debugging purposes, we also have
\verb=\leftmargini= set to
\texttt{\the\leftmargini}, and we have
\verb=\leftmarginii= set to
\texttt{\the\leftmarginii}.

This is a \hilitebox{\texttt{quotation}}
environment with one paragraph, following a fairly
short paragraph of prose (notice that the
\texttt{quotation} paragraphs \emph{are}
indented):

\begin{singlespace}
\color{darkblue}
\begin{verbatim}
\begin{quotation}
    \color{blue}
    Algebra is concerned with manipulation in
    \emph{time}, and geometry is concerned with
    \emph{space}. These are two orthogonal aspects
    of the world, and they represent two different
    points of view in mathematics.  Thus the
    argument or dialogue between mathematicians in
    the past about the relative importance of
    geometry and algebra represents something very
    fundamental.
    \hfill
    \emph{Sir Michael Atiyah}
    % Mathematics in the 20$^{th}$ century
    % NTM {\bf 10}(1--3) 25--39 (September 2002)
    % http://dx.doi.org/10.1007/BF03033096
\end{quotation}
\end{verbatim}
\end{singlespace}
%
\begin{quotation}
    \color{blue}
    Algebra is concerned with manipulation in
    \emph{time}, and geometry is concerned with
    \emph{space}. These are two orthogonal aspects
    of the world, and they represent two different
    points of view in mathematics.  Thus the
    argument or dialogue between mathematicians in
    the past about the relative importance of
    geometry and algebra represents something very
    fundamental.
    \hfill
    \emph{Sir Michael Atiyah}
    % Mathematics in the 20$^{th}$ century
    % NTM {\bf 10}(1--3) 25--39 (September 2002)
    % http://dx.doi.org/10.1007/BF03033096
\end{quotation}

This is a \hilitebox{\texttt{quotation}} environment with three
paragraphs, following a fairly short paragraph of prose:

\begin{quotation}
    \color{utahred}
    Algebra is concerned with manipulation in
    \emph{time}, and geometry is concerned with
    \emph{space}. These are two orthogonal aspects
    of the world, and they represent two different
    points of view in mathematics.  Thus the
    argument or dialogue between mathematicians in
    the past about the relative importance of
    geometry and algebra represents something very
    fundamental.
    \hfill
    \emph{Sir Michael Atiyah}

    Algebra is concerned with manipulation in
    \emph{time}, and geometry is concerned with
    \emph{space}. These are two orthogonal aspects
    of the world, and they represent two different
    points of view in mathematics.  Thus the
    argument or dialogue between mathematicians in
    the past about the relative importance of
    geometry and algebra represents something very
    fundamental.
    \hfill
    \emph{Sir Michael Atiyah}

    Algebra is concerned with manipulation in
    \emph{time}, and geometry is concerned with
    \emph{space}. These are two orthogonal aspects
    of the world, and they represent two different
    points of view in mathematics.  Thus the
    argument or dialogue between mathematicians in
    the past about the relative importance of
    geometry and algebra represents something very
    fundamental.
    \hfill
    \emph{Sir Michael Atiyah}
\end{quotation}

%% \showthe \baselineskip

Now all following text should be back in
double-spaced mode, and just go on and on and on
and on and on and on and on and on and on and on
and on and on and on and on and on and on and on
and on and on and on and on and on and on and on
and on and on and on and on and on and on and on
and on and on and on and on and on and on and on
and on and on and on and on and on and on and on
and on and on and on and on and on and on and on
and on and on and on and on and on and on and on
and on and on and on and on and on and on and on
and on and on and on and on and on and on and on
and on and on and on and on \ldots{}.

%% \showthe \baselineskip

Now all following text should be back in
double-spaced mode, and just go on and on and on
and on and on and on and on and on and on and on
and on and on and on and on and on and on and on
and on and on and on and on and on and on and on
and on and on and on and on and on and on and on
and on and on and on and on and on and on and on
and on and on and on and on and on and on and on
and on and on and on and on and on and on and on
and on and on and on and on and on and on and on
and on and on and on and on and on and on and on
and on and on and on and on and on and on and on
and on and on and on and on \ldots{}.

%%% \showthe \baselineskip
}
%%%
%%% just before the \maintext macro that begins the body of the thesis,
%%% and starts chapter numbering.
%%%
%%% [16-Mar-2016]
%%% ====================================================================

\input{rgb.sty}

\definecolor{utahred} {rgb} {0.8, 0.0, 0.0} % official definition for University of Utah Printing Services

\doublespace

%% \showthe \baselineskip

In this section, we use color in several places.
The \verb=\colorbox= command takes two arguments
--- a named color and text to be in black on a
background of that color --- and sets the text in
a box with a small margin of width \verb=\fboxsep=
(set to \texttt{\the\fboxsep} in this document).

Here, we want a tighter colored box that has a
fixed height, and is independent of letter shape.
We set the margin to zero inside a group so that
the change is purely local, and so that height and
depth of the line are not increased over what they
would be if the colored box were not used.  We
prefix a \TeX{} \verb=\strut= to the user-supplied
text, because that command expands to a zero-width
box of the height and depth of parentheses, which,
in most fonts, delimit the extent of letter
shapes.

\begin{verbatim}
\newcommand {\hilitebox} [1] {{\fboxsep = 0pt\colorbox{pink}{\strut #1}}}
\end{verbatim}

\newcommand {\hilitebox} [1] {{\fboxsep = 0pt\colorbox{pink}{\strut #1}}}

Here is a fragment from the first chapter in another
thesis, set in \emph{emphasized text} to distinguish it
from the rest of this section:

\begin{itshape}
    In light of the known results, the consistency
    of empirical semivariogram and related
    estimators is widely considered a settled
    matter.  For example, Lahiri, Lee, and Cressie
    \cite{Lahiri:2002:ADA} state:

    \begin{quote}
        The simpler and more commonly used
        nonparametric estimators of the variogram,
        such as the method of moments estimator of
        Matheron (1962) and its robustified
        versions due to Cressie and Hawkins (1980)
        have many desirable properties like,
        unbiasedness, consistency, etc. \ldots
    \end{quote}
    \noindent
    Regarding a kernel estimator of the covariance
    function, Hall and Patil
    \cite{Hall:1994:PNE} remarked:
    \begin{quote}
        It is not difficult to see that if, as $ n
        $ increases, the points $ t_i $ become
        increasingly dense in each bounded subset
        of $ \mathbb{R}^d $, then the bandwidth $
        h $ may be chosen so that $ \check \rho(t)
        \to \rho(t) $ as $ n \to \infty $, for
        each $ t \in \mathbb{R}^d $.
    \end{quote}
    However, in order to be true, such statements
    would need to be qualified by many assumptions
    on the random field as well as on the
    observation locations. We will see in
    \S2.3 that even for well-behaved
    random fields (e.g., $\rho^*$-mixing Gaussian
    random fields), it is not enough to assume
    that the observation locations become
    increasingly dense in each bounded subset; a
    stronger assumption must be made to ensure
    that the observation locations do not become
    denser in one region too much faster than in
    others.
\end{itshape}

The text before the previous paragraph contained
two \texttt{quote} environments separated by a
line of prose.  Here are some more tests of both
kinds of \LaTeX{} environments for showing text
written by someone else.

This is a \hilitebox{\texttt{quote}} environment
with one short line, following a fairly short
paragraph of prose (in this, and following
examples, the text is explicitly colored with a
command like \verb=\color{purple}= inside the
environment before the text):
%
\begin{singlespace}
\color{darkblue}
\begin{verbatim}
\begin{quote}
    \color{purple}
    14 March 2016 is $\pi \approx 3.1416$ day in funny notation.
    \hfill \emph{Web news reports}
\end{quote}
\end{verbatim}
\end{singlespace}
%
\begin{quote}
    \color{purple}
    14 March 2016 is $\pi \approx 3.1416$ day in funny notation.
    \hfill \emph{Web news reports}
\end{quote}

This is a \hilitebox{\texttt{quote}} environment
with three short lines, each a separate paragraph,
following a fairly short paragraph of prose.

\begin{singlespace}
\color{darkblue}
\begin{verbatim}
\begin{quote}
    \color{forestgreen}
    14 March 2016 is $\pi \approx 3.1416$ day in funny notation.
    \hfill \emph{Web news reports}

    14 March 2016 is $\pi \approx 3.1416$ day in funny notation.
    \hfill \emph{Web news reports}

    14 March 2016 is $\pi \approx 3.1416$ day in funny notation.
    \hfill \emph{Web news reports}
\end{quote}
\end{verbatim}
\end{singlespace}
%
\begin{quote}
    \color{forestgreen}
    14 March 2016 is $\pi \approx 3.1416$ day in funny notation.
    \hfill \emph{Web news reports}

    14 March 2016 is $\pi \approx 3.1416$ day in funny notation.
    \hfill \emph{Web news reports}

    14 March 2016 is $\pi \approx 3.1416$ day in funny notation.
    \hfill \emph{Web news reports}
\end{quote}

Here is another example, this time with separate
colors for each paragraph:

\begin{singlespace}
\color{darkblue}
\begin{verbatim}
\begin{quote}
    \color{darkkhaki}
    14 March 2016 is $\pi \approx 3.1416$ day in funny notation.
    \hfill \emph{Web news reports}

    \color{darkmagenta}
    14 March 2016 is $\pi \approx 3.1416$ day in funny notation.
    \hfill \emph{Web news reports}

    \color{darkcyan}
    14 March 2016 is $\pi \approx 3.1416$ day in funny notation.
    \hfill \emph{Web news reports}

    \color{darkorange}
    14 March 2016 is $\pi \approx 3.1416$ day in funny notation.
    14 March 2016 is $\pi \approx 3.1416$ day in funny notation.
    14 March 2016 is $\pi \approx 3.1416$ day in funny notation.
    \linebreak
    \strut
    \hfill \emph{Web news reports}
\end{quote}
\end{verbatim}
\end{singlespace}
%
\begin{quote}
    \color{darkkhaki}
    14 March 2016 is $\pi \approx 3.1416$ day in funny notation.
    \hfill \emph{Web news reports}

    \color{darkmagenta}
    14 March 2016 is $\pi \approx 3.1416$ day in funny notation.
    \hfill \emph{Web news reports}

    \color{darkcyan}
    14 March 2016 is $\pi \approx 3.1416$ day in funny notation.
    \hfill \emph{Web news reports}

    \color{darkorange}
    14 March 2016 is $\pi \approx 3.1416$ day in funny notation.
    14 March 2016 is $\pi \approx 3.1416$ day in funny notation.
    14 March 2016 is $\pi \approx 3.1416$ day in funny notation.
    \linebreak
    \strut
    \hfill \emph{Web news reports}
\end{quote}

Notice that \hilitebox{\texttt{quote}} paragraphs
are \emph{not} indented, but the environment
itself \emph{is} indented on the left and right by
the value of \verb=\leftmargin= (set to
\texttt{\the\leftmargin} in this document, which
should be identical to \verb=2.5em=, where
\verb=1em= = \texttt{\dimen0 = 1em \the\dimen0}).

For debugging purposes, we also have
\verb=\leftmargini= set to
\texttt{\the\leftmargini}, and we have
\verb=\leftmarginii= set to
\texttt{\the\leftmarginii}.

This is a \hilitebox{\texttt{quotation}}
environment with one paragraph, following a fairly
short paragraph of prose (notice that the
\texttt{quotation} paragraphs \emph{are}
indented):

\begin{singlespace}
\color{darkblue}
\begin{verbatim}
\begin{quotation}
    \color{blue}
    Algebra is concerned with manipulation in
    \emph{time}, and geometry is concerned with
    \emph{space}. These are two orthogonal aspects
    of the world, and they represent two different
    points of view in mathematics.  Thus the
    argument or dialogue between mathematicians in
    the past about the relative importance of
    geometry and algebra represents something very
    fundamental.
    \hfill
    \emph{Sir Michael Atiyah}
    % Mathematics in the 20$^{th}$ century
    % NTM {\bf 10}(1--3) 25--39 (September 2002)
    % http://dx.doi.org/10.1007/BF03033096
\end{quotation}
\end{verbatim}
\end{singlespace}
%
\begin{quotation}
    \color{blue}
    Algebra is concerned with manipulation in
    \emph{time}, and geometry is concerned with
    \emph{space}. These are two orthogonal aspects
    of the world, and they represent two different
    points of view in mathematics.  Thus the
    argument or dialogue between mathematicians in
    the past about the relative importance of
    geometry and algebra represents something very
    fundamental.
    \hfill
    \emph{Sir Michael Atiyah}
    % Mathematics in the 20$^{th}$ century
    % NTM {\bf 10}(1--3) 25--39 (September 2002)
    % http://dx.doi.org/10.1007/BF03033096
\end{quotation}

This is a \hilitebox{\texttt{quotation}} environment with three
paragraphs, following a fairly short paragraph of prose:

\begin{quotation}
    \color{utahred}
    Algebra is concerned with manipulation in
    \emph{time}, and geometry is concerned with
    \emph{space}. These are two orthogonal aspects
    of the world, and they represent two different
    points of view in mathematics.  Thus the
    argument or dialogue between mathematicians in
    the past about the relative importance of
    geometry and algebra represents something very
    fundamental.
    \hfill
    \emph{Sir Michael Atiyah}

    Algebra is concerned with manipulation in
    \emph{time}, and geometry is concerned with
    \emph{space}. These are two orthogonal aspects
    of the world, and they represent two different
    points of view in mathematics.  Thus the
    argument or dialogue between mathematicians in
    the past about the relative importance of
    geometry and algebra represents something very
    fundamental.
    \hfill
    \emph{Sir Michael Atiyah}

    Algebra is concerned with manipulation in
    \emph{time}, and geometry is concerned with
    \emph{space}. These are two orthogonal aspects
    of the world, and they represent two different
    points of view in mathematics.  Thus the
    argument or dialogue between mathematicians in
    the past about the relative importance of
    geometry and algebra represents something very
    fundamental.
    \hfill
    \emph{Sir Michael Atiyah}
\end{quotation}

%% \showthe \baselineskip

Now all following text should be back in
double-spaced mode, and just go on and on and on
and on and on and on and on and on and on and on
and on and on and on and on and on and on and on
and on and on and on and on and on and on and on
and on and on and on and on and on and on and on
and on and on and on and on and on and on and on
and on and on and on and on and on and on and on
and on and on and on and on and on and on and on
and on and on and on and on and on and on and on
and on and on and on and on and on and on and on
and on and on and on and on and on and on and on
and on and on and on and on \ldots{}.

%% \showthe \baselineskip

Now all following text should be back in
double-spaced mode, and just go on and on and on
and on and on and on and on and on and on and on
and on and on and on and on and on and on and on
and on and on and on and on and on and on and on
and on and on and on and on and on and on and on
and on and on and on and on and on and on and on
and on and on and on and on and on and on and on
and on and on and on and on and on and on and on
and on and on and on and on and on and on and on
and on and on and on and on and on and on and on
and on and on and on and on and on and on and on
and on and on and on and on \ldots{}.

%%% \showthe \baselineskip
}
%%%
%%% just before the \maintext macro that begins the body of the thesis,
%%% and starts chapter numbering.
%%%
%%% [16-Mar-2016]
%%% ====================================================================

\input{rgb.sty}

\definecolor{utahred} {rgb} {0.8, 0.0, 0.0} % official definition for University of Utah Printing Services

\doublespace

%% \showthe \baselineskip

In this section, we use color in several places.
The \verb=\colorbox= command takes two arguments
--- a named color and text to be in black on a
background of that color --- and sets the text in
a box with a small margin of width \verb=\fboxsep=
(set to \texttt{\the\fboxsep} in this document).

Here, we want a tighter colored box that has a
fixed height, and is independent of letter shape.
We set the margin to zero inside a group so that
the change is purely local, and so that height and
depth of the line are not increased over what they
would be if the colored box were not used.  We
prefix a \TeX{} \verb=\strut= to the user-supplied
text, because that command expands to a zero-width
box of the height and depth of parentheses, which,
in most fonts, delimit the extent of letter
shapes.

\begin{verbatim}
\newcommand {\hilitebox} [1] {{\fboxsep = 0pt\colorbox{pink}{\strut #1}}}
\end{verbatim}

\newcommand {\hilitebox} [1] {{\fboxsep = 0pt\colorbox{pink}{\strut #1}}}

Here is a fragment from the first chapter in another
thesis, set in \emph{emphasized text} to distinguish it
from the rest of this section:

\begin{itshape}
    In light of the known results, the consistency
    of empirical semivariogram and related
    estimators is widely considered a settled
    matter.  For example, Lahiri, Lee, and Cressie
    \cite{Lahiri:2002:ADA} state:

    \begin{quote}
        The simpler and more commonly used
        nonparametric estimators of the variogram,
        such as the method of moments estimator of
        Matheron (1962) and its robustified
        versions due to Cressie and Hawkins (1980)
        have many desirable properties like,
        unbiasedness, consistency, etc. \ldots
    \end{quote}
    \noindent
    Regarding a kernel estimator of the covariance
    function, Hall and Patil
    \cite{Hall:1994:PNE} remarked:
    \begin{quote}
        It is not difficult to see that if, as $ n
        $ increases, the points $ t_i $ become
        increasingly dense in each bounded subset
        of $ \mathbb{R}^d $, then the bandwidth $
        h $ may be chosen so that $ \check \rho(t)
        \to \rho(t) $ as $ n \to \infty $, for
        each $ t \in \mathbb{R}^d $.
    \end{quote}
    However, in order to be true, such statements
    would need to be qualified by many assumptions
    on the random field as well as on the
    observation locations. We will see in
    \S2.3 that even for well-behaved
    random fields (e.g., $\rho^*$-mixing Gaussian
    random fields), it is not enough to assume
    that the observation locations become
    increasingly dense in each bounded subset; a
    stronger assumption must be made to ensure
    that the observation locations do not become
    denser in one region too much faster than in
    others.
\end{itshape}

The text before the previous paragraph contained
two \texttt{quote} environments separated by a
line of prose.  Here are some more tests of both
kinds of \LaTeX{} environments for showing text
written by someone else.

This is a \hilitebox{\texttt{quote}} environment
with one short line, following a fairly short
paragraph of prose (in this, and following
examples, the text is explicitly colored with a
command like \verb=\color{purple}= inside the
environment before the text):
%
\begin{singlespace}
\color{darkblue}
\begin{verbatim}
\begin{quote}
    \color{purple}
    14 March 2016 is $\pi \approx 3.1416$ day in funny notation.
    \hfill \emph{Web news reports}
\end{quote}
\end{verbatim}
\end{singlespace}
%
\begin{quote}
    \color{purple}
    14 March 2016 is $\pi \approx 3.1416$ day in funny notation.
    \hfill \emph{Web news reports}
\end{quote}

This is a \hilitebox{\texttt{quote}} environment
with three short lines, each a separate paragraph,
following a fairly short paragraph of prose.

\begin{singlespace}
\color{darkblue}
\begin{verbatim}
\begin{quote}
    \color{forestgreen}
    14 March 2016 is $\pi \approx 3.1416$ day in funny notation.
    \hfill \emph{Web news reports}

    14 March 2016 is $\pi \approx 3.1416$ day in funny notation.
    \hfill \emph{Web news reports}

    14 March 2016 is $\pi \approx 3.1416$ day in funny notation.
    \hfill \emph{Web news reports}
\end{quote}
\end{verbatim}
\end{singlespace}
%
\begin{quote}
    \color{forestgreen}
    14 March 2016 is $\pi \approx 3.1416$ day in funny notation.
    \hfill \emph{Web news reports}

    14 March 2016 is $\pi \approx 3.1416$ day in funny notation.
    \hfill \emph{Web news reports}

    14 March 2016 is $\pi \approx 3.1416$ day in funny notation.
    \hfill \emph{Web news reports}
\end{quote}

Here is another example, this time with separate
colors for each paragraph:

\begin{singlespace}
\color{darkblue}
\begin{verbatim}
\begin{quote}
    \color{darkkhaki}
    14 March 2016 is $\pi \approx 3.1416$ day in funny notation.
    \hfill \emph{Web news reports}

    \color{darkmagenta}
    14 March 2016 is $\pi \approx 3.1416$ day in funny notation.
    \hfill \emph{Web news reports}

    \color{darkcyan}
    14 March 2016 is $\pi \approx 3.1416$ day in funny notation.
    \hfill \emph{Web news reports}

    \color{darkorange}
    14 March 2016 is $\pi \approx 3.1416$ day in funny notation.
    14 March 2016 is $\pi \approx 3.1416$ day in funny notation.
    14 March 2016 is $\pi \approx 3.1416$ day in funny notation.
    \linebreak
    \strut
    \hfill \emph{Web news reports}
\end{quote}
\end{verbatim}
\end{singlespace}
%
\begin{quote}
    \color{darkkhaki}
    14 March 2016 is $\pi \approx 3.1416$ day in funny notation.
    \hfill \emph{Web news reports}

    \color{darkmagenta}
    14 March 2016 is $\pi \approx 3.1416$ day in funny notation.
    \hfill \emph{Web news reports}

    \color{darkcyan}
    14 March 2016 is $\pi \approx 3.1416$ day in funny notation.
    \hfill \emph{Web news reports}

    \color{darkorange}
    14 March 2016 is $\pi \approx 3.1416$ day in funny notation.
    14 March 2016 is $\pi \approx 3.1416$ day in funny notation.
    14 March 2016 is $\pi \approx 3.1416$ day in funny notation.
    \linebreak
    \strut
    \hfill \emph{Web news reports}
\end{quote}

Notice that \hilitebox{\texttt{quote}} paragraphs
are \emph{not} indented, but the environment
itself \emph{is} indented on the left and right by
the value of \verb=\leftmargin= (set to
\texttt{\the\leftmargin} in this document, which
should be identical to \verb=2.5em=, where
\verb=1em= = \texttt{\dimen0 = 1em \the\dimen0}).

For debugging purposes, we also have
\verb=\leftmargini= set to
\texttt{\the\leftmargini}, and we have
\verb=\leftmarginii= set to
\texttt{\the\leftmarginii}.

This is a \hilitebox{\texttt{quotation}}
environment with one paragraph, following a fairly
short paragraph of prose (notice that the
\texttt{quotation} paragraphs \emph{are}
indented):

\begin{singlespace}
\color{darkblue}
\begin{verbatim}
\begin{quotation}
    \color{blue}
    Algebra is concerned with manipulation in
    \emph{time}, and geometry is concerned with
    \emph{space}. These are two orthogonal aspects
    of the world, and they represent two different
    points of view in mathematics.  Thus the
    argument or dialogue between mathematicians in
    the past about the relative importance of
    geometry and algebra represents something very
    fundamental.
    \hfill
    \emph{Sir Michael Atiyah}
    % Mathematics in the 20$^{th}$ century
    % NTM {\bf 10}(1--3) 25--39 (September 2002)
    % http://dx.doi.org/10.1007/BF03033096
\end{quotation}
\end{verbatim}
\end{singlespace}
%
\begin{quotation}
    \color{blue}
    Algebra is concerned with manipulation in
    \emph{time}, and geometry is concerned with
    \emph{space}. These are two orthogonal aspects
    of the world, and they represent two different
    points of view in mathematics.  Thus the
    argument or dialogue between mathematicians in
    the past about the relative importance of
    geometry and algebra represents something very
    fundamental.
    \hfill
    \emph{Sir Michael Atiyah}
    % Mathematics in the 20$^{th}$ century
    % NTM {\bf 10}(1--3) 25--39 (September 2002)
    % http://dx.doi.org/10.1007/BF03033096
\end{quotation}

This is a \hilitebox{\texttt{quotation}} environment with three
paragraphs, following a fairly short paragraph of prose:

\begin{quotation}
    \color{utahred}
    Algebra is concerned with manipulation in
    \emph{time}, and geometry is concerned with
    \emph{space}. These are two orthogonal aspects
    of the world, and they represent two different
    points of view in mathematics.  Thus the
    argument or dialogue between mathematicians in
    the past about the relative importance of
    geometry and algebra represents something very
    fundamental.
    \hfill
    \emph{Sir Michael Atiyah}

    Algebra is concerned with manipulation in
    \emph{time}, and geometry is concerned with
    \emph{space}. These are two orthogonal aspects
    of the world, and they represent two different
    points of view in mathematics.  Thus the
    argument or dialogue between mathematicians in
    the past about the relative importance of
    geometry and algebra represents something very
    fundamental.
    \hfill
    \emph{Sir Michael Atiyah}

    Algebra is concerned with manipulation in
    \emph{time}, and geometry is concerned with
    \emph{space}. These are two orthogonal aspects
    of the world, and they represent two different
    points of view in mathematics.  Thus the
    argument or dialogue between mathematicians in
    the past about the relative importance of
    geometry and algebra represents something very
    fundamental.
    \hfill
    \emph{Sir Michael Atiyah}
\end{quotation}

%% \showthe \baselineskip

Now all following text should be back in
double-spaced mode, and just go on and on and on
and on and on and on and on and on and on and on
and on and on and on and on and on and on and on
and on and on and on and on and on and on and on
and on and on and on and on and on and on and on
and on and on and on and on and on and on and on
and on and on and on and on and on and on and on
and on and on and on and on and on and on and on
and on and on and on and on and on and on and on
and on and on and on and on and on and on and on
and on and on and on and on and on and on and on
and on and on and on and on \ldots{}.

%% \showthe \baselineskip

Now all following text should be back in
double-spaced mode, and just go on and on and on
and on and on and on and on and on and on and on
and on and on and on and on and on and on and on
and on and on and on and on and on and on and on
and on and on and on and on and on and on and on
and on and on and on and on and on and on and on
and on and on and on and on and on and on and on
and on and on and on and on and on and on and on
and on and on and on and on and on and on and on
and on and on and on and on and on and on and on
and on and on and on and on and on and on and on
and on and on and on and on \ldots{}.

%%% \showthe \baselineskip
}
\maintext       % Start normal page numbering: parts and chapters follow.
\pagestyle{headings} % NEW for sample-thesis-6


% chapters of my dissertation
\include{chapters/1-intro}
\include{chapters/2-background}
\include{chapters/3-framework}
\include{chapters/4-example}
\include{chapters/5-security}
\include{chapters/6-worksheet}
\include{chapters/7-reflect}
\include{chapters/8-discuss}
\include{chapters/9-conclude}


% appendices
\numberofappendices = 3 % set 0 for none, else number of appendices
\appendix % label all sections A, B, etc.
\include{tex/11-appendix-methods}
\include{tex/12-appendix-worksheets}
\include{tex/13-appendix-evaluation}
%\nocite{*}

% print out refs chapter according to thesis template (from uuthesis2e.cls)
\newpage
  \thispagestyle{empty}%
  \addcontentsline{toc}{chapter}{REFERENCES}%
  \mainheading{REFERENCES}%
  \par\removelastskip\singlespace\par\removelastskip% GBG Oct 1993
  \fixmainheadingSKIP

% show references
\printbibliography[heading=none]

\end{document}
